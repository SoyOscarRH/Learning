% =======================================================
% =======         HEADER FOR DOCUMENT        ============
% =======================================================
    
    % *********  SPECIFIC FOR THIS BOOK  ********
    \def\ProjectAuthorLink{https://github.com/CompilandoConocimiento}
    \def\ProjectNameLink{\ProjectAuthorLink/CellularAutomata}    
    

    % *********   DOCUMENT ITSELF   **************
    \documentclass[12pt, fleqn]{report}                             %Type of doc and size of font and left equations
    \usepackage[margin=1.2in]{geometry}                             %Margins and Geometry pacakge
    \usepackage{ifthen}                                             %Allow simple programming using if - then
    \usepackage[hidelinks]{hyperref}                                %Allow to create hiperlinks and Fuck Firefox
    \usepackage{pdfpages}                                           %Allow us 'import' PDF's
    \hypersetup{pageanchor=false}                                   %Solve 'double page 1' warnings in build :v
    \setlength{\parindent}{0pt}                                     %Eliminate ugly indentation
    \author{Oscar Andrés Rosas}                                     %Who I am

    % *********   LANGUAJE    *****************
    \usepackage[spanish]{babel}                                     %Please allow me to type in spanish
    \usepackage[utf8]{inputenc}                                     %Lets use UFT-8
    \usepackage[T1]{fontenc}                                        %Allow for better font support
    \usepackage{textcmds}                                           %Allow us to use quoutes
    \usepackage{changepage}                                         %Allow us to use identate paragraphs
    \usepackage{anyfontsize}                                        %All the sizes for fonts wiiiii!

    % *********   MATH AND HIS STYLE  *********
    \usepackage{ntheorem, amsmath, amssymb, amsfonts}               %All fucking math, I want all!
    \usepackage{mathrsfs, mathtools, empheq}                        %All fucking math, I want all!
    \usepackage{cancel}                                             %Negate symbol
    \usepackage{centernot}                                          %Allow me to negate a symbol
    \decimalpoint                                                   %Use decimal point

    % *********   GRAPHICS AND IMAGES *********
    \usepackage{graphicx}                                           %Allow to create graphics
    \usepackage{float}                                              %For images
    \usepackage{wrapfig}                                            %Allow to create images
    \graphicspath{ {Graphics/} }                                    %Where are the images :D

    % *********   LISTS AND TABLES ***********
    \usepackage{listings, listingsutf8}                             %We will be using code here
    \usepackage[inline]{enumitem}                                   %We will need to enumarate
    \usepackage{tasks}                                              %Horizontal lists
    \usepackage{longtable}                                          %Lets make tables awesome
    \usepackage{booktabs}                                           %Lets make tables awesome
    \usepackage{tabularx}                                           %Lets make tables awesome
    \usepackage{multirow}                                           %Lets make tables awesome
    \usepackage{multicol}                                           %Create multicolumns

    % *********   REMOVE SOME ERRORS **********
    \hbadness=10000                                                 %Ignore \vbox and \hbox warings
    \hfuzz=\maxdimen\newdimen\hfuzz                                 %Ignore \vbox and \hbox warings

    % *********   HEADERS AND FOOTERS ********
    \usepackage{fancyhdr}                                           %Lets make awesome headers/footers
    \pagestyle{fancy}                                               %Lets make awesome headers/footers
    \setlength{\headheight}{16pt}                                   %Top line
    \setlength{\parskip}{0.5em}                                     %Top line
    \renewcommand{\footrulewidth}{0.5pt}                            %Bottom line

    \lhead {                                                        %Left Header
        \hyperlink{chapter.\arabic{chapter}}                        %Make a link to the current chapter
        {\normalsize{\textsc{\nouppercase{\leftmark}}}}             %And fot it put the name
    }

    \rhead {                                                        %Right Header
        \hyperlink{section.\arabic{chapter}.\arabic{section}}       %Make a link to the current chapter
            {\footnotesize{\textsc{\nouppercase{\rightmark}}}}      %And fot it put the name
    }

    \rfoot{\textsc{\small{\hyperref[sec:Index]{Ve al Índice}}}}     %This will always be a footer  

    \fancyfoot[L]{                                                  %Algoritm for a changing footer
        \ifthenelse{\isodd{\value{page}}}                           %IF ODD PAGE:
            {\href{https://SoyOscarRH.github.io/}                   %DO THIS:
                {\footnotesize                                      %Send the page
                    {\textsc{Oscar Andrés Rosas}}}}                 %Send the page
            {\href{https://compilandoconocimiento.com}              %ELSE DO THIS: 
                {\footnotesize                                      %Send the author
                    {\textsc{Compilando Conocimiento}}}}            %Send the author
    }
    
    
% =======================================================
% ===================   COMMANDS    =====================
% =======================================================

    % =========================================
    % =======   NEW ENVIRONMENTS   ============
    % =========================================
    \newenvironment{Indentation}[1][0.75em]                         %Use: \begin{Inde...}[Num]...\end{Inde...}
        {\begin{adjustwidth}{#1}{}}                                 %If you dont put nothing i will use 0.75 em
        {\end{adjustwidth}}                                         %This indentate a paragraph
    
    \newenvironment{SmallIndentation}[1][0.75em]                    %Use: The same that we upper one, just 
        {\begin{adjustwidth}{#1}{}\begin{footnotesize}}             %footnotesize size of letter by default
        {\end{footnotesize}\end{adjustwidth}}                       %that's it
    
    \def \Eq {equation}                                             %Stupid Visual studio error
    \newenvironment{MultiLineEquation}[1]                           %Use: To create MultiLine equations
        {\begin{\Eq}\begin{alignedat}{#1}}                          %Use: \begin{Multi..}{Num. de Columnas}
        {\end{alignedat}\end{\Eq}}                                  %And.. that's it!
    
    \newenvironment{MultiLineEquation*}[1]                          %Use: To create MultiLine equations
        {\begin{\Eq*}\begin{alignedat}{#1}}                         %Use: \begin{Multi..}{Num. de Columnas}
        {\end{alignedat}\end{\Eq*}}                                 %And.. that's it!

    \newenvironment{largeEq} {\begingroup \large}{\endgroup}        %Make eq bigger
    \newenvironment{LargeEq} {\begingroup \Large}{\endgroup}        %Make eq bigger
    \newenvironment{HugeEq} {\begingroup \Huge}{\endgroup}          %Make eq bigger!

    % =========================================
    % == GENERAL TEXT & SYMBOLS ENVIRONMENTS ==
    % =========================================
    
    % =====  TEXT  ======================
    \newcommand \Quote              {\qq}                           %Use: \Quote to use quotes
    \newcommand \Over               {\overline}                     %Use: \Bar to use just for short
    \newcommand \ForceNewLine       {$\Space$\\}                    %Use it in theorems for example
    \newcommand \ForceColumnBreak   {\vfill\null\columnbreak}       %Use only in multicols
    \newcommand \Link[2] {\underline{\texttt{\href{#1}{#2}}}}       %Use a link

    % =====  SPACES  ====================
    \DeclareMathOperator \Space     {\quad}                         %Use: \Space for a cool mega space
    \DeclareMathOperator \MegaSpace {\quad \quad}                   %Use: \MegaSpace for a cool mega mega space
    \DeclareMathOperator \MiniSpace {\;}                            %Use: \Space for a cool mini space
    
    % =====  MATH TEXT  =================
    \newcommand \Such           {\MiniSpace | \MiniSpace}           %Use: \Such like in sets
    \newcommand \Also           {\MiniSpace \text{y} \MiniSpace}    %Use: \Also so it's look cool
    \newcommand \Remember[1]    {\Space\text{\scriptsize{#1}}}      %Use: \Remember so it's look cool
    
    % =====  THEOREMS: IN SPANISH :0  ===
    \newtheorem{Theorem}        {Teorema}[section]                  %Use: \begin{Theorem}[Name]\label{Nombre}...
    \newtheorem{Corollary}      {Colorario}[Theorem]                %Use: \begin{Corollary}[Name]\label{Nombre}...
    \newtheorem{Lemma}[Theorem] {Lemma}                             %Use: \begin{Lemma}[Name]\label{Nombre}...
    \newtheorem{Definition}     {Definición}[section]               %Use: \begin{Definition}[Name]\label{Nombre}...
    \theoremstyle{break}                                            %THEOREMS START 1 SPACE AFTER Fuck!

    % =====  LOGIC  =====================
    \newcommand \lIff    {\leftrightarrow}                          %Use: \lIff for logic iff
    \newcommand \lEqual  {\MiniSpace \Leftrightarrow \MiniSpace}    %Use: \lEqual for a logic double arrow
    \newcommand \lInfire {\MiniSpace \Rightarrow \MiniSpace}        %Use: \lInfire for a logic infire
    \newcommand \lLongTo {\longrightarrow}                          %Use: \lLongTo for a long arrow
    \newcommand \lAnd    {\land}                                    %Use: \lAnd ^
    \newcommand \lOr     {\lor}                                     %Use: \lOr or symbol
    \newcommand \lNot    {\neg}                                     %Use: \lNot for negation

    % =====  FAMOUS SETS  ===============
    \DeclareMathOperator \Naturals     {\mathbb{N}}                 %Use: \Naturals por Notation
    \DeclareMathOperator \Primes       {\mathbb{P}}                 %Use: \Primes por Notation
    \DeclareMathOperator \Integers     {\mathbb{Z}}                 %Use: \Integers por Notation
    \DeclareMathOperator \Racionals    {\mathbb{Q}}                 %Use: \Racionals por Notation
    \DeclareMathOperator \Reals        {\mathbb{R}}                 %Use: \Reals por Notation
    \DeclareMathOperator \Complexs     {\mathbb{C}}                 %Use: \Complex por Notation
    \DeclareMathOperator \GenericField {\mathbb{F}}                 %Use: \GenericField por Notation
    \DeclareMathOperator \VectorSet    {\mathbb{V}}                 %Use: \VectorSet por Notation
    \DeclareMathOperator \SubVectorSet {\mathbb{W}}                 %Use: \SubVectorSet por Notation
    \DeclareMathOperator \Polynomials  {\mathbb{P}}                 %Use: \Polynomials por Notation
    \DeclareMathOperator \VectorSpace  {\VectorSet_{\GenericField}} %Use: \VectorSpace por Notation
    \DeclareMathOperator \LinealTransformation {\mathcal{T}}        %Use: \LinealTransformation for a cool T
    \DeclareMathOperator \LinTrans      {\mathcal{T}}               %Use: \LinTrans for a cool T
    \DeclareMathOperator \Laplace       {\mathcal{L}}               %Use: \LinTrans for a cool T

    % =====  CONTAINERS   ===============
    \newcommand{\Set}[1]            {\left\{ \; #1 \; \right\}}     %Use: \Set {Info} for INTELLIGENT space 
    \newcommand{\bigSet}[1]         {\big\{  \; #1 \; \big\}}       %Use: \bigSet  {Info} for space 
    \newcommand{\BigSet}[1]         {\Big\{  \; #1 \; \Big\}}       %Use: \BigSet  {Info} for space 
    \newcommand{\biggSet}[1]        {\bigg\{ \; #1 \; \bigg\}}      %Use: \biggSet {Info} for space 
    \newcommand{\BiggSet}[1]        {\Bigg\{ \; #1 \; \Bigg\}}      %Use: \BiggSet {Info} for space 
        
    \newcommand{\Wrap}[1]           {\left( #1 \right)}             %Use: \Wrap {Info} for INTELLIGENT space
    \newcommand{\bigWrap}[1]        {\big( \; #1 \; \big)}          %Use: \bigBrackets  {Info} for space 
    \newcommand{\BigWrap}[1]        {\Big( \; #1 \; \Big)}          %Use: \BigBrackets  {Info} for space 
    \newcommand{\biggWrap}[1]       {\bigg( \; #1 \; \bigg)}        %Use: \biggBrackets {Info} for space 
    \newcommand{\BiggWrap}[1]       {\Bigg( \; #1 \; \Bigg)}        %Use: \BiggBrackets {Info} for space 

    \newcommand{\Brackets}[1]       {\left[ #1 \right]}             %Use: \Brackets {Info} for INTELLIGENT space
    \newcommand{\bigBrackets}[1]    {\big[ \; #1 \; \big]}          %Use: \bigBrackets  {Info} for space 
    \newcommand{\BigBrackets}[1]    {\Big[ \; #1 \; \Big]}          %Use: \BigBrackets  {Info} for space 
    \newcommand{\biggBrackets}[1]   {\bigg[ \; #1 \; \bigg]}        %Use: \biggBrackets {Info} for space 
    \newcommand{\BiggBrackets}[1]   {\Bigg[ \; #1 \; \Bigg]}        %Use: \BiggBrackets {Info} for space 

    \newcommand{\Generate}[1]   {\left\langle #1 \right\rangle}     %Use: \Generate {Info} <>
    \newcommand{\Floor}[1]      {\left \lfloor #1 \right \rfloor}   %Use: \Floor {Info} for floor 
    \newcommand{\Ceil}[1]       {\left \lceil #1 \right \rceil }    %Use: \Ceil {Info} for ceil
    
    % =====  BETTERS MATH COMMANDS   =====
    \newcommand{\pfrac}[2]      {\Wrap{\dfrac{#1}{#2}}}             %Use: Put fractions in parentesis
    \newcommand{\Sum}           {\displaystyle \sum}                %Use: Sum to big sum
    \newcommand{\Int}           {\displaystyle \int}                %Use: Sum to big integral


    % =========================================
    % ====   LINEAL ALGEBRA & VECTORS    ======
    % =========================================

    % ===== UNIT VECTORS  ================
    \newcommand{\hati}      {\hat{\imath}}                           %Use: \hati for unit vector    
    \newcommand{\hatj}      {\hat{\jmath}}                           %Use: \hatj for unit vector    
    \newcommand{\hatk}      {\hat{k}}                                %Use: \hatk for unit vector

    % ===== MAGNITUDE  ===================
    \newcommand{\abs}[1]    {\left\lvert #1 \right\lvert}           %Use: \abs{expression} for |x|
    \newcommand{\Abs}[1]    {\left\lVert #1 \right\lVert}           %Use: \Abs{expression} for ||x||
    \newcommand{\Mag}[1]    {\left| #1 \right|}                     %Use: \Mag {Info} 
    
    \newcommand{\bVec}[1]   {\mathbf{#1}}                           %Use for bold type of vector
    \newcommand{\lVec}[1]   {\overrightarrow{#1}}                   %Use for a long arrow over a vector
    \newcommand{\uVec}[1]   {\mathbf{\hat{#1}}}                     %Use: Unitary Vector Example: $\uVec{i}

    % ===== FN LINEAL TRANSFORMATION  ====
    \newcommand{\FnLinTrans}[1]{\mathcal{T}\Wrap{#1}}               %Use: \FnLinTrans for a cool T
    \newcommand{\VecLinTrans}[1]{\mathcal{T}\pVector{#1}}           %Use: \LinTrans for a cool T
    \newcommand{\FnLinealTransformation}[1]{\mathcal{T}\Wrap{#1}}   %Use: \FnLinealTransformation

    % ===== ALL FOR DOT PRODUCT  =========
    \makeatletter                                                   %WTF! IS THIS
    \newcommand*\dotP{\mathpalette\dotP@{.5}}                       %Use: \dotP for dot product
    \newcommand*\dotP@[2] {\mathbin {                               %WTF! IS THIS            
        \vcenter{\hbox{\scalebox{#2}{$\m@th#1\bullet$}}}}           %WTF! IS THIS
    }                                                               %WTF! IS THIS
    \makeatother                                                    %WTF! IS THIS

    % === WRAPPERS FOR COLUMN VECTOR ===
    \newcommand{\pVector}[1]                                        %Use: \pVector {Matrix Notation} use parentesis
        { \ensuremath{\begin{pmatrix}#1\end{pmatrix}} }             %Example: \pVector{a\\b\\c} or \pVector{a&b&c} 
    \newcommand{\lVector}[1]                                        %Use: \lVector {Matrix Notation} use a abs 
        { \ensuremath{\begin{vmatrix}#1\end{vmatrix}} }             %Example: \lVector{a\\b\\c} or \lVector{a&b&c} 
    \newcommand{\bVector}[1]                                        %Use: \bVector {Matrix Notation} use a brackets 
        { \ensuremath{\begin{bmatrix}#1\end{bmatrix}} }             %Example: \bVector{a\\b\\c} or \bVector{a&b&c} 
    \newcommand{\Vector}[1]                                         %Use: \Vector {Matrix Notation} no parentesis
        { \ensuremath{\begin{matrix}#1\end{matrix}} }               %Example: \Vector{a\\b\\c} or \Vector{a&b&c}

    % === MAKE MATRIX BETTER  =========
    \makeatletter                                                   %Example: \begin{matrix}[cc|c]
    \renewcommand*\env@matrix[1][*\c@MaxMatrixCols c] {             %WTF! IS THIS
        \hskip -\arraycolsep                                        %WTF! IS THIS
        \let\@ifnextchar\new@ifnextchar                             %WTF! IS THIS
        \array{#1}                                                  %WTF! IS THIS
    }                                                               %WTF! IS THIS
    \makeatother                                                    %WTF! IS THIS
    
    \newcommand{\adotP}[2] {\left< #1, #2 \right> }                 %Use for <x, y>
    \newcommand{\wdotP}[2] {\Wrap{ #1, #2 } }                       %Use for (x, y)
    \newcommand{\cdotP}[2] {\Wrap{ #1 \dotP #2 } }                  %Use for (x * y)


    % =========================================
    % =======   FAMOUS FUNCTIONS   ============
    % =========================================

    % == TRIGONOMETRIC FUNCTIONS  ====
    \newcommand{\Cos}[1] {\cos\Wrap{#1}}                            %Simple wrappers
    \newcommand{\Sin}[1] {\sin\Wrap{#1}}                            %Simple wrappers
    \newcommand{\Tan}[1] {tan\Wrap{#1}}                             %Simple wrappers
    
    \newcommand{\Sec}[1] {sec\Wrap{#1}}                             %Simple wrappers
    \newcommand{\Csc}[1] {csc\Wrap{#1}}                             %Simple wrappers
    \newcommand{\Cot}[1] {cot\Wrap{#1}}                             %Simple wrappers

    % === COMPLEX ANALYSIS TRIG ======
    \newcommand \Cis[1]  {\Cos{#1} + i \Sin{#1}}                    %Use: \Cis for cos(x) + i sin(x)
    \newcommand \pCis[1] {\Wrap{\Cis{#1}}}                          %Use: \pCis for the same with parantesis
    \newcommand \bCis[1] {\Brackets{\Cis{#1}}}                      %Use: \bCis for the same with Brackets


    % =========================================
    % ===========     CALCULUS     ============
    % =========================================

    % ====== TRANSFORMS =============
    \newcommand{\FourierT}[1]   {\mathscr{F} \left\{ #1 \right\} }  %Use: \FourierT {Funtion}
    \newcommand{\InvFourierT}[1]{\mathscr{F}^{-1}\left\{#1\right\}} %Use: \InvFourierT {Funtion}

    % ====== DERIVATIVES ============
    \newcommand \MiniDerivate[1][x]   {\dfrac{d}{d #1}}             %Use: \MiniDerivate[var] for simple use [var]
    \newcommand \Derivate[2]          {\dfrac{d \; #1}{d #2}}       %Use: \Derivate [f(x)][x]
    \newcommand \MiniUpperDerivate[2] {\dfrac{d^{#2}}{d#1^{#2}}}    %Mini Derivate High Orden Derivate -- [x][pow]
    \newcommand \UpperDerivate[3] {\dfrac{d^{#3} \; #1}{d#2^{#3}}}  %Complete High Orden Derivate -- [f(x)][x][pow]
    
    \newcommand \MiniPartial[1][x] {\dfrac{\partial}{\partial #1}}  %Use: \MiniDerivate for simple use [var]
    \newcommand \Partial[2] {\dfrac{\partial \; #1}{\partial #2}}   %Complete Partial Derivate -- [f(x)][x]
    \newcommand \MiniUpperPartial[2]                                %Mini Derivate High Orden Derivate -- [x][pow] 
        {\dfrac{\partial^{#2}}{\partial #1^{#2}}}                   %Mini Derivate High Orden Derivate
    \newcommand \UpperPartial[3]                                    %Complete High Orden Derivate -- [f(x)][x][pow]
        {\dfrac{\partial^{#3} \; #1}{\partial#2^{#3}}}              %Use: \UpperDerivate for simple use

    \DeclareMathOperator \Evaluate  {\Big|}                         %Use: \Evaluate por Notation

    % ====== INTEGRALS ============
    \newcommand{\inftyInt} {\int_{-\infty}^{\infty}}                %Use: \inftyInt for simple integrants
    
        
% =======================================================
% ===========      COLOR: MATERIAL DESIGN     ===========
% =======================================================

    % =====  COLORS ==================
    \definecolor{RedMD}{HTML}{F44336}                               %Use: Color :D        
    \definecolor{Red100MD}{HTML}{FFCDD2}                            %Use: Color :D        
    \definecolor{Red200MD}{HTML}{EF9A9A}                            %Use: Color :D        
    \definecolor{Red300MD}{HTML}{E57373}                            %Use: Color :D        
    \definecolor{Red700MD}{HTML}{D32F2F}                            %Use: Color :D 

    \definecolor{PurpleMD}{HTML}{9C27B0}                            %Use: Color :D        
    \definecolor{Purple100MD}{HTML}{E1BEE7}                         %Use: Color :D        
    \definecolor{Purple200MD}{HTML}{EF9A9A}                         %Use: Color :D        
    \definecolor{Purple300MD}{HTML}{BA68C8}                         %Use: Color :D        
    \definecolor{Purple700MD}{HTML}{7B1FA2}                         %Use: Color :D 

    \definecolor{IndigoMD}{HTML}{3F51B5}                            %Use: Color :D        
    \definecolor{Indigo100MD}{HTML}{C5CAE9}                         %Use: Color :D        
    \definecolor{Indigo200MD}{HTML}{9FA8DA}                         %Use: Color :D        
    \definecolor{Indigo300MD}{HTML}{7986CB}                         %Use: Color :D        
    \definecolor{Indigo700MD}{HTML}{303F9F}                         %Use: Color :D 

    \definecolor{BlueMD}{HTML}{2196F3}                              %Use: Color :D        
    \definecolor{Blue100MD}{HTML}{BBDEFB}                           %Use: Color :D        
    \definecolor{Blue200MD}{HTML}{90CAF9}                           %Use: Color :D        
    \definecolor{Blue300MD}{HTML}{64B5F6}                           %Use: Color :D        
    \definecolor{Blue700MD}{HTML}{1976D2}                           %Use: Color :D        
    \definecolor{Blue900MD}{HTML}{0D47A1}                           %Use: Color :D  

    \definecolor{CyanMD}{HTML}{00BCD4}                              %Use: Color :D        
    \definecolor{Cyan100MD}{HTML}{B2EBF2}                           %Use: Color :D        
    \definecolor{Cyan200MD}{HTML}{80DEEA}                           %Use: Color :D        
    \definecolor{Cyan300MD}{HTML}{4DD0E1}                           %Use: Color :D        
    \definecolor{Cyan700MD}{HTML}{0097A7}                           %Use: Color :D        
    \definecolor{Cyan900MD}{HTML}{006064}                           %Use: Color :D 

    \definecolor{TealMD}{HTML}{009688}                              %Use: Color :D        
    \definecolor{Teal100MD}{HTML}{B2DFDB}                           %Use: Color :D        
    \definecolor{Teal200MD}{HTML}{80CBC4}                           %Use: Color :D        
    \definecolor{Teal300MD}{HTML}{4DB6AC}                           %Use: Color :D        
    \definecolor{Teal700MD}{HTML}{00796B}                           %Use: Color :D        
    \definecolor{Teal900MD}{HTML}{004D40}                           %Use: Color :D 

    \definecolor{GreenMD}{HTML}{4CAF50}                             %Use: Color :D        
    \definecolor{Green100MD}{HTML}{C8E6C9}                          %Use: Color :D        
    \definecolor{Green200MD}{HTML}{A5D6A7}                          %Use: Color :D        
    \definecolor{Green300MD}{HTML}{81C784}                          %Use: Color :D        
    \definecolor{Green700MD}{HTML}{388E3C}                          %Use: Color :D        
    \definecolor{Green900MD}{HTML}{1B5E20}                          %Use: Color :D

    \definecolor{AmberMD}{HTML}{FFC107}                             %Use: Color :D        
    \definecolor{Amber100MD}{HTML}{FFECB3}                          %Use: Color :D        
    \definecolor{Amber200MD}{HTML}{FFE082}                          %Use: Color :D        
    \definecolor{Amber300MD}{HTML}{FFD54F}                          %Use: Color :D        
    \definecolor{Amber700MD}{HTML}{FFA000}                          %Use: Color :D        
    \definecolor{Amber900MD}{HTML}{FF6F00}                          %Use: Color :D

    \definecolor{OrangeMD}{HTML}{ff9800}                            %Use: Color :D        
    \definecolor{Orange100MD}{HTML}{ffe0b2}                         %Use: Color :D        
    \definecolor{Orange200MD}{HTML}{ffcc80}                         %Use: Color :D        
    \definecolor{Orange300MD}{HTML}{ffb74d}                         %Use: Color :D        
    \definecolor{Orange700MD}{HTML}{fb8c00}                         %Use: Color :D        
    \definecolor{Orange900MD}{HTML}{ef6c00}                         %Use: Color :D

    \definecolor{BlueGreyMD}{HTML}{607D8B}                          %Use: Color :D        
    \definecolor{BlueGrey100MD}{HTML}{CFD8DC}                       %Use: Color :D        
    \definecolor{BlueGrey200MD}{HTML}{B0BEC5}                       %Use: Color :D        
    \definecolor{BlueGrey300MD}{HTML}{90A4AE}                       %Use: Color :D        
    \definecolor{BlueGrey700MD}{HTML}{455A64}                       %Use: Color :D        
    \definecolor{BlueGrey900MD}{HTML}{263238}                       %Use: Color :D        

    \definecolor{DeepPurpleMD}{HTML}{673AB7}                        %Use: Color :D

    \definecolor{SolarizedBase}{HTML}{fdf6e3}                       %Use: Color :D
    \definecolor{SolarizedFont}{HTML}{073642}                       %Use: Color :D

    % =====  ENVIRONMENT ==============
    \newcommand{\Color}[2]{\textcolor{#1}{#2}}                      %Simple color environment
    \newenvironment{ColorText}[1]                                   %Use: \begin{ColorText}
        { \leavevmode\color{#1}\ignorespaces }                      %That's is!


% =======================================================
% ===========           CODE EDITING          ===========
% =======================================================

    \newcommand{\fontCode}        { \ttfamily\bfseries }            %Use: \fontCode for font
    \newcommand{\fontCodeTiny}    { \fontCode\tiny }                %Sizes
    \newcommand{\fontCodeFoot}    { \fontCode\footnotesize }        %Sizes
    \newcommand{\fontCodeScript}  { \fontCode\scriptsize }          %Sizes
    \newcommand{\fontCodeCostume} { \fontCode\fontsize{10}{7} }     %Sizes
   

    % =====  CODE EDITOR =============
    \lstdefinestyle{CompilandoStyle} {                              %This is Code Style
        backgroundcolor     = \color{BlueGrey900MD},                %Background Color  
        basicstyle          = \fontCodeTiny\color{white},           %Style of text
        commentstyle        = \color{BlueGrey200MD},                %Comment style
        stringstyle         = \color{Green300MD},                   %String style
        keywordstyle        = \color{Blue300MD},                    %keywords style
        numberstyle         = \tiny\color{TealMD},                  %Size of a number
        frame               = none,                                 %Adds a frame around the code
        breakatwhitespace   = true,                                 %Style   
        breaklines          = true,                                 %Style   
        showstringspaces    = false,                                %Hate those spaces                  
        breaklines          = true,                                 %Style                   
        keepspaces          = true,                                 %Style                   
        numbers             = left,                                 %Style                   
        numbersep           = 10pt,                                 %Style 
        xleftmargin         = \parindent,                           %Style 
        tabsize             = 4,                                    %Style
        inputencoding       = utf8/latin1                           %Allow me to use special chars
    }

    % =====  CODE EDITOR =============
    \lstdefinestyle{CompilandoStylePurity} {                        %This is Code Style
        backgroundcolor     = \color{white},                        %Background Color  
        basicstyle          = \fontCodeTiny\color{BlueGrey900MD},   %Style of text
        commentstyle        = \color{Green300MD},                   %Comment style
        stringstyle         = \color{Teal700MD},                    %String style
        keywordstyle        = \color{Blue700MD},                    %keywords style
        numberstyle         = \tiny\color{TealMD},                  %Size of a number
        frame               = none,                                 %Adds a frame around the code
        breakatwhitespace   = true,                                 %Style   
        breaklines          = true,                                 %Style   
        showstringspaces    = false,                                %Hate those spaces                  
        breaklines          = true,                                 %Style                   
        keepspaces          = true,                                 %Style                   
        numbers             = left,                                 %Style                   
        numbersep           = 11pt,                                 %Style 
        xleftmargin         = \parindent,                           %Style 
        tabsize             = 4,                                    %Style
        inputencoding       = utf8/latin1                           %Allow me to use special chars
    }

    % =====  CODE EDITOR =============
    \lstdefinestyle{CompilandoStyleSolarized} {                     %This is Code Style
        backgroundcolor     = \color{SolarizedBase},                %Background Color  
        basicstyle          = \fontCodeFoot\color{SolarizedFont},   %Style of text
        commentstyle        = \color{Green300MD},                   %Comment style
        stringstyle         = \color{Teal700MD},                    %String style
        keywordstyle        = \color{Blue700MD},                    %keywords style
        numberstyle         = \tiny\color{TealMD},                  %Size of a number
        frame               = none,                                 %Adds a frame around the code
        breakatwhitespace   = true,                                 %Style   
        breaklines          = true,                                 %Style   
        showstringspaces    = false,                                %Hate those spaces                  
        breaklines          = true,                                 %Style                   
        keepspaces          = true,                                 %Style                   
        numbers             = none,                                 %Style                   
        tabsize             = 4,                                    %Style
        inputencoding       = utf8/latin1                           %Allow me to use special chars
    }
 
    \lstset{style = CompilandoStyleSolarized}                          %Use this style



% =====================================================
% ============        COVER PAGE       ================
% =====================================================
\begin{document}
\begin{titlepage}
    
    % ============ TITLE PAGE STYLE  ================
    \definecolor{TitlePageColor}{cmyk}{1,.60,0,.40}                 %Simple colors
    \definecolor{ColorSubtext}{cmyk}{1,.50,0,.10}                   %Simple colors
    \newgeometry{left=0.25\textwidth}                               %Defines an Offset
    \pagecolor{TitlePageColor}                                      %Make it this Color to page
    \color{white}                                                   %General things should be white

    % ===== MAKE SOME SPACE =========
    \vspace                                                         %Give some space
    \baselineskip                                                   %But we need this to up command

    % ============ NAME OF THE PROJECT  ============
    \makebox[0pt][l]{\rule{1.3\textwidth}{3pt}}                     %Make a cool line
    
    \href{https://compilandoconocimiento.com}                       %Link to project
    {\textbf{\textsc{\Huge Instituto Politécnico Nacional, Escuela Superior de Cómputo}}}\\[2.7cm]      %Name of project   

    % ============ NAME OF THE BOOK  ===============
    \href{\ProjectNameLink}                                         %Link to Author
    {\fontsize{45}{52}\selectfont \textbf{Autómata Celulares}}\\[0.5cm] %Name of the book
    \textcolor{ColorSubtext}{\textsc{\Huge Sistemas Complejos}}     %Name of the general theme
    
    \vfill                                                          %Fill the space
    
    % ============ NAME OF THE AUTHOR  =============
    \href{\ProjectAuthorLink}                                       %Link to Author
    {\LARGE \textsf{Oscar Andrés Rosas Hernandez}}                  %Author

    % ===== MAKE SOME SPACE =========
    \vspace                                                         %Give some space
    \baselineskip                                                   %But we need this to up command
    
    {\large \textsf{\today}}                                        %Date

\end{titlepage}


% =====================================================
% ==========      RESTORE TO DOCUMENT      ============
% =====================================================
\restoregeometry                                                    %Restores the geometry
\nopagecolor                                                        %Use to restore the color to white




% =====================================================
% ========                INDICE              =========
% =====================================================
\tableofcontents{}
\label{sec:Index}

\clearpage


\part{Marco Teórico}
\clearpage

    \chapter{Autómata celulares}

        \clearpage
        \section{Definición}

        Un autómata celular es un sistema dinámico discreto que consiste en una red regular
        de autómatas (celdas) de estado finito que cambian sus estados dependiendo de los estados de sus
        vecinos (y del mismo), de acuerdo con una función de transferencia. 
        
        Todas las células cambian su estado simultáneamente usando la misma regla de actualización. 
        El proceso se repite en pasos de tiempo discretos. 
        Resulta que con reglas de actualización sorprendentemente simples se pueden producir 
        dinámicas extremadamente complejas como en el famoso Juego de la vida de John Conway.  \cite{Kari}
        
      \subsection{Caracteristicas}
      \begin{itemize}
        \item Son discretos tanto en tiempo como en espacio
        \item Son homogeneos tanto en tiempo como en espacio 
        (la misma regla es aplicada a todas las celulas al mismo tiempo)
        \item Sus interacciones son locales
      \end{itemize}

      Para especificar una autómata celular, debemos especificar los siguientes 
      elementos (algunos de los cuales pueden ser claros por el contexto):
      \begin{itemize}
        \item La dimensión $d \in Z^+$,
        \item El conjunto de estados finitos $S$
        \item Una vecindad N de celdas
        \item La función de activación $f: S^m \to S$
      \end{itemize}

      Por lo tanto, definimos formalmente a un autómata celular 
      correspondiente como la 4-tupla $A = (d, S, N, f)$.

      \section{Autómata celulares elementales}

      Los autómatas elementales son autómatas celulares unidimensionales con dos estados y una
      vecindad de radio 1: $d = 1$, $S = \Set{0, 1}$, $N = (-1, 0, 1)$.
      
      Se diferencian entre sí solo en la elección de la regla $f$. 
      Hay 256 autómata elementales.

      \section{Clases de Wolfram}

      S.Wolfram trabajó en los años 80 con los autómatas elementales y basándose en observaciones empíricas 
      de su comportamiento en configuraciones iniciales aleatorias, las clasificó en cuatro clases.

      Estas se conocen como clases Wolfram. 
      Las definiciones no son matemáticamente rigurosas, y desde entonces se han propuesto 
      clasificaciones más precisas.

      Wolfram definió las clases de la siguiente manera:
      \begin{itemize}
        \item (W1): Casi todas las configuraciones iniciales conducen a la misma configuración uniforme de punto fijo,
        \item (W2): Casi todas las configuraciones iniciales conducen a una configuración que se repite periódicamente,
        \item (W3): Casi todas las configuraciones iniciales conducen a un comportamiento esencialmente aleatorio,
        \item (W4): Surgen estructuras localizadas con interacciones complejas.
      \end{itemize}

      \cite{Kari}

      Es importante recalcar que Wolfram las definia de tal manera que la regla 2 estuviera
      contenida en al regla 1 y asi con la regla 3 y la 2 y la 4 y la 3.

      Wolfram conjeturó que los autómatas celulares de esa clase (W4) son computacionalmente universales.
      (cosa que se ha probado para la regla 110).

  \chapter{La reducción de 256 reglas a solo 88}

      En su libro A new Kind of Science, S.Wolfram \cite{Wolfram} logra demostrar de manera bastante trivial
      que las 356 reglas se pueden reducir a solo 88 reglas irreducibles, para hacerlo se basa en una idea
      bastante sencilla:

      Tienes una regla $n$ entonces:
      \begin{itemize}
        \item La regla que cambia todos los unos por cero no cambia la naturaleza de la regla
        \item La regla que cambia derecha por izquierda no cambia la naturaleza de la regla
        \item La regla que cambia las dos anteriores al mismo tiempo no cambia la naturaleza de la regla
      \end{itemize}

      Asi podemos hacer grupos de 4, este programa en C++ nos permite obtener las clases de manera sencilla:

      Estas son las clases equivalentes:
      \begin{itemize}
        \item 0, 255, 0, 255
        \item 1, 127, 1, 127
        \item 2, 191, 16, 247
        \item 3, 63, 17, 119
        \item 4, 223, 4, 223
        \item 5, 95, 5, 95
        \item 6, 159, 20, 215
        \item 7, 31, 21, 87
        \item 8, 239, 64, 253
        \item 9, 111, 65, 125
        \item 10, 175, 80, 245
        \item 11, 47, 81, 117
        \item 12, 207, 68, 221
        \item 13, 79, 69, 93
        \item 14, 143, 84, 213
        \item 15, 15, 85, 85
        \item 16, 247, 2, 191
        \item 17, 119, 3, 63
        \item 18, 183, 18, 183
        \item 19, 55, 19, 55
        \item 20, 215, 6, 159
        \item 21, 87, 7, 31
        \item 22, 151, 22, 151
        \item 23, 23, 23, 23
        \item 24, 231, 66, 189
        \item 25, 103, 67, 61
        \item 26, 167, 82, 181
        \item 27, 39, 83, 53
        \item 28, 199, 70, 157
        \item 29, 71, 71, 29
        \item 30, 135, 86, 149
        \item 31, 7, 87, 21
        \item 32, 251, 32, 251
        \item 33, 123, 33, 123
        \item 34, 187, 48, 243
        \item 35, 59, 49, 115
        \item 36, 219, 36, 219
        \item 37, 91, 37, 91
        \item 38, 155, 52, 211
        \item 39, 27, 53, 83
        \item 40, 235, 96, 249
        \item 41, 107, 97, 121
        \item 42, 171, 112, 241
        \item 43, 43, 113, 113
        \item 44, 203, 100, 217
        \item 45, 75, 101, 89
        \item 46, 139, 116, 209
        \item 47, 11, 117, 81
        \item 48, 243, 34, 187
        \item 49, 115, 35, 59
        \item 50, 179, 50, 179
        \item 51, 51, 51, 51
        \item 52, 211, 38, 155
        \item 53, 83, 39, 27
        \item 54, 147, 54, 147
        \item 55, 19, 55, 19
        \item 56, 227, 98, 185
        \item 57, 99, 99, 57
        \item 58, 163, 114, 177
        \item 59, 35, 115, 49
        \item 60, 195, 102, 153
        \item 61, 67, 103, 25
        \item 62, 131, 118, 145
        \item 63, 3, 119, 17
        \item 64, 253, 8, 239
        \item 65, 125, 9, 111
        \item 66, 189, 24, 231
        \item 67, 61, 25, 103
        \item 68, 221, 12, 207
        \item 69, 93, 13, 79
        \item 70, 157, 28, 199
        \item 71, 29, 29, 71
        \item 72, 237, 72, 237
        \item 73, 109, 73, 109
        \item 74, 173, 88, 229
        \item 75, 45, 89, 101
        \item 76, 205, 76, 205
        \item 77, 77, 77, 77
        \item 78, 141, 92, 197
        \item 79, 13, 93, 69
        \item 80, 245, 10, 175
        \item 81, 117, 11, 47
        \item 82, 181, 26, 167
        \item 83, 53, 27, 39
        \item 84, 213, 14, 143
        \item 85, 85, 15, 15
        \item 86, 149, 30, 135
        \item 87, 21, 31, 7
        \item 88, 229, 74, 173
        \item 89, 101, 75, 45
        \item 90, 165, 90, 165
        \item 91, 37, 91, 37
        \item 92, 197, 78, 141
        \item 93, 69, 79, 13
        \item 94, 133, 94, 133
        \item 95, 5, 95, 5
        \item 96, 249, 40, 235
        \item 97, 121, 41, 107
        \item 98, 185, 56, 227
        \item 99, 57, 57, 99
        \item 100, 217, 44, 203
        \item 101, 89, 45, 75
        \item 102, 153, 60, 195
        \item 103, 25, 61, 67
        \item 104, 233, 104, 233
        \item 105, 105, 105, 105
        \item 106, 169, 120, 225
        \item 107, 41, 121, 97
        \item 108, 201, 108, 201
        \item 109, 73, 109, 73
        \item 110, 137, 124, 193
        \item 111, 9, 125, 65
        \item 112, 241, 42, 171
        \item 113, 113, 43, 43
        \item 114, 177, 58, 163
        \item 115, 49, 59, 35
        \item 116, 209, 46, 139
        \item 117, 81, 47, 11
        \item 118, 145, 62, 131
        \item 119, 17, 63, 3
        \item 120, 225, 106, 169
        \item 121, 97, 107, 41
        \item 122, 161, 122, 161
        \item 123, 33, 123, 33
        \item 124, 193, 110, 137
        \item 125, 65, 111, 9
        \item 126, 129, 126, 129
        \item 127, 1, 127, 1
        \item 128, 254, 128, 254
        \item 129, 126, 129, 126
        \item 130, 190, 144, 246
        \item 131, 62, 145, 118
        \item 132, 222, 132, 222
        \item 133, 94, 133, 94
        \item 134, 158, 148, 214
        \item 135, 30, 149, 86
        \item 136, 238, 192, 252
        \item 137, 110, 193, 124
        \item 138, 174, 208, 244
        \item 139, 46, 209, 116
        \item 140, 206, 196, 220
        \item 141, 78, 197, 92
        \item 142, 142, 212, 212
        \item 143, 14, 213, 84
        \item 144, 246, 130, 190
        \item 145, 118, 131, 62
        \item 146, 182, 146, 182
        \item 147, 54, 147, 54
        \item 148, 214, 134, 158
        \item 149, 86, 135, 30
        \item 150, 150, 150, 150
        \item 151, 22, 151, 22
        \item 152, 230, 194, 188
        \item 153, 102, 195, 60
        \item 154, 166, 210, 180
        \item 155, 38, 211, 52
        \item 156, 198, 198, 156
        \item 157, 70, 199, 28
        \item 158, 134, 214, 148
        \item 159, 6, 215, 20
        \item 160, 250, 160, 250
        \item 161, 122, 161, 122
        \item 162, 186, 176, 242
        \item 163, 58, 177, 114
        \item 164, 218, 164, 218
        \item 165, 90, 165, 90
        \item 166, 154, 180, 210
        \item 167, 26, 181, 82
        \item 168, 234, 224, 248
        \item 169, 106, 225, 120
        \item 170, 170, 240, 240
        \item 171, 42, 241, 112
        \item 172, 202, 228, 216
        \item 173, 74, 229, 88
        \item 174, 138, 244, 208
        \item 175, 10, 245, 80
        \item 176, 242, 162, 186
        \item 177, 114, 163, 58
        \item 178, 178, 178, 178
        \item 179, 50, 179, 50
        \item 180, 210, 166, 154
        \item 181, 82, 167, 26
        \item 182, 146, 182, 146
        \item 183, 18, 183, 18
        \item 184, 226, 226, 184
        \item 185, 98, 227, 56
        \item 186, 162, 242, 176
        \item 187, 34, 243, 48
        \item 188, 194, 230, 152
        \item 189, 66, 231, 24
        \item 190, 130, 246, 144
        \item 191, 2, 247, 16
        \item 192, 252, 136, 238
        \item 193, 124, 137, 110
        \item 194, 188, 152, 230
        \item 195, 60, 153, 102
        \item 196, 220, 140, 206
        \item 197, 92, 141, 78
        \item 198, 156, 156, 198
        \item 199, 28, 157, 70
        \item 200, 236, 200, 236
        \item 201, 108, 201, 108
        \item 202, 172, 216, 228
        \item 203, 44, 217, 100
        \item 204, 204, 204, 204
        \item 205, 76, 205, 76
        \item 206, 140, 220, 196
        \item 207, 12, 221, 68
        \item 208, 244, 138, 174
        \item 209, 116, 139, 46
        \item 210, 180, 154, 166
        \item 211, 52, 155, 38
        \item 212, 212, 142, 142
        \item 213, 84, 143, 14
        \item 214, 148, 158, 134
        \item 215, 20, 159, 6
        \item 216, 228, 202, 172
        \item 217, 100, 203, 44
        \item 218, 164, 218, 164
        \item 219, 36, 219, 36
        \item 220, 196, 206, 140
        \item 221, 68, 207, 12
        \item 222, 132, 222, 132
        \item 223, 4, 223, 4
        \item 224, 248, 168, 234
        \item 225, 120, 169, 106
        \item 226, 184, 184, 226
        \item 227, 56, 185, 98
        \item 228, 216, 172, 202
        \item 229, 88, 173, 74
        \item 230, 152, 188, 194
        \item 231, 24, 189, 66
        \item 232, 232, 232, 232
        \item 233, 104, 233, 104
        \item 234, 168, 248, 224
        \item 235, 40, 249, 96
        \item 236, 200, 236, 200
        \item 237, 72, 237, 72
        \item 238, 136, 252, 192
        \item 239, 8, 253, 64
        \item 240, 240, 170, 170
        \item 241, 112, 171, 42
        \item 242, 176, 186, 162
        \item 243, 48, 187, 34
        \item 244, 208, 174, 138
        \item 245, 80, 175, 10
        \item 246, 144, 190, 130
        \item 247, 16, 191, 2
        \item 248, 224, 234, 168
        \item 249, 96, 235, 40
        \item 250, 160, 250, 160
        \item 251, 32, 251, 32
        \item 252, 192, 238, 136
        \item 253, 64, 239, 8
        \item 254, 128, 254, 128
        \item 255, 0, 255, 0
      \end{itemize}


      Eligiendo la regla mas pequeña de cada clase podemos general el conjunto de reglas que vamos a
      analizar: \{0, 1, 2, 3, 4, 5, 6, 7, 8, 9, 10, 11, 12, 13, 14, 15, 18, 19, 22, 23, 24, 25, 26, 27, 28, 29, 
      30, 32, 33, 34, 35, 36, 37, 38, 40, 41, 42, 43, 44, 45, 46, 50, 51, 54, 56, 57, 58, 60, 62, 72, 73, 
      74, 76, 77, 78, 90, 94, 104, 105, 106, 108, 110, 122, 126, 128, 130, 132, 134, 136, 138, 140, 142, 
      146, 150, 152, 154, 156, 160, 162, 164, 168, 170, 172, 178, 184, 200, 204, 232\}.

      El código realizado fue:
      \lstinputlisting[language=C++, gobble=6]{Code/Equivalence.cpp}

      Compilado con:
      \begin{lstlisting}[language=C++, gobble=6]
        g++ -std=c++17 Equivalence.cpp && ./a.out
      \end{lstlisting}

      Es trivial comprobar que el codigo funciona comparando con los resultados
      obtenido por Wolfram \cite{Wolfram}.

  \chapter{Mi clasificación de los autómatas celulares}
      \section{W1}
        \begin{itemize}
          \item Regla 160
        \end{itemize}

      \section{W2}
        \begin{itemize}
          \item Regla 108
        \end{itemize}

      \section{W3}
        \begin{itemize}
          \item Regla 126
        \end{itemize}

      \section{W4}
        \begin{itemize}
          \item Regla 110
          \item Regla 54
        \end{itemize}

\begin{thebibliography}{10}

  \bibitem{Kari} 
      \textit{Cellular Automata}. 
      Jarkko Kari, Spring 2013 \\
      \url{https://www.cs.tau.ac.il/~nachumd/models/CA.pdf}

  \bibitem{Wolfram} 
      \textit{A New Kind of Science}. 
      Wolfram Stephen, 2002 \\

\end{thebibliography}



\end{document}