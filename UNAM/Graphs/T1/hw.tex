
 % *****************************************************************************************************
% ****************************              SIMPLE EXAMPLE             ********************************
% *****************************************************************************************************

% =======================================================
% =======         HEADER FOR DOCUMENT        ============
% =======================================================
    
    % *********  SPECIFIC FOR THIS BOOK  ********
    \def\ProjectAuthorLink{https://github.com/CompilandoConocimiento}
    \def\ProjectNameLink{\ProjectAuthorLink/RandomProject}    
    

    % *********   DOCUMENT ITSELF   **************
    \documentclass[journal,12pt,onecolumn]{IEEEtran}                %Type of doc and size of font and left equations
    \usepackage{ifthen}                                             %Allow simple programming using if - then
    \usepackage[hidelinks]{hyperref}                                %Allow to create hiperlinks and Fuck Firefox
    \usepackage{pdfpages}                                           %Allow us 'import' PDF's
    \hypersetup{pageanchor=false}                                   %Solve 'double page 1' warnings in build :v
    \setlength{\parindent}{0pt}                                     %Eliminate ugly indentation
    \author{Oscar Andrés Rosas}                                     %Who I am

    % *********   LANGUAJE    *****************
    \usepackage[spanish]{babel}                                     %Please allow me to type in spanish
    \usepackage[utf8]{inputenc}                                     %Lets use UFT-8
    \usepackage[T1]{fontenc}                                        %Allow for better font support
    \usepackage{textcmds}                                           %Allow us to use quoutes
    \usepackage{changepage}                                         %Allow us to use identate paragraphs
    \usepackage{anyfontsize}                                        %All the sizes for fonts wiiiii!

    % *********   MATH AND HIS STYLE  *********
    \usepackage{ntheorem, amsmath, amssymb, amsfonts}               %All fucking math, I want all!
    \usepackage{mathrsfs, mathtools, empheq}                        %All fucking math, I want all!
    \usepackage{cancel}                                             %Negate symbol
    \usepackage{centernot}                                          %Allow me to negate a symbol
    \decimalpoint                                                   %Use decimal point

    % *********   GRAPHICS AND IMAGES *********
    \usepackage{graphicx}                                           %Allow to create graphics
    \usepackage{float}                                              %For images
    \usepackage{wrapfig}                                            %Allow to create images
    \graphicspath{ {Graphics/} }                                    %Where are the images :D

    % *********   LISTS AND TABLES ***********
    \usepackage{listings, listingsutf8}                             %We will be using code here
    \usepackage[inline]{enumitem}                                   %We will need to enumarate
    \usepackage{tasks}                                              %Horizontal lists
    \usepackage{longtable}                                          %Lets make tables awesome
    \usepackage{booktabs}                                           %Lets make tables awesome
    \usepackage{tabularx}                                           %Lets make tables awesome
    \usepackage{multirow}                                           %Lets make tables awesome
    \usepackage{multicol}                                           %Create multicolumns
    \usepackage{lscape}

    % *********   REMOVE SOME ERRORS **********
    \hbadness=10000                                                 %Ignore \vbox and \hbox warings
    \hfuzz=\maxdimen\newdimen\hfuzz                                 %Ignore \vbox and \hbox warings

    % *********   HEADERS AND FOOTERS ********
    \usepackage{fancyhdr}                                           %Lets make awesome headers/footers
    \pagestyle{fancy}                                               %Lets make awesome headers/footers
    \setlength{\headheight}{16pt}                                   %Top line
    \setlength{\parskip}{0.5em}                                     %Top line
    \renewcommand{\footrulewidth}{0.5pt}                            %Bottom line

    
% =======================================================
% ===================   COMMANDS    =====================
% =======================================================

    % =========================================
    % =======   NEW ENVIRONMENTS   ============
    % =========================================
    \newenvironment{Indentation}[1][0.75em]                         %Use: \begin{Inde...}[Num]...\end{Inde...}
        {\begin{adjustwidth}{#1}{}}                                 %If you dont put nothing i will use 0.75 em
        {\end{adjustwidth}}                                         %This indentate a paragraph
    
    \newenvironment{SmallIndentation}[1][0.75em]                    %Use: The same that we upper one, just 
        {\begin{adjustwidth}{#1}{}\begin{footnotesize}}             %footnotesize size of letter by default
        {\end{footnotesize}\end{adjustwidth}}                       %that's it
    
    \def \Eq {equation}                                             %Stupid Visual studio error
    \newenvironment{MultiLineEquation}[1]                           %Use: To create MultiLine equations
        {\begin{\Eq}\begin{alignedat}{#1}}                          %Use: \begin{Multi..}{Num. de Columnas}
        {\end{alignedat}\end{\Eq}}                                  %And.. that's it!
    
    \newenvironment{MultiLineEquation*}[1]                          %Use: To create MultiLine equations
        {\begin{\Eq*}\begin{alignedat}{#1}}                         %Use: \begin{Multi..}{Num. de Columnas}
        {\end{alignedat}\end{\Eq*}}                                 %And.. that's it!

    \newenvironment{largeEq} {\begingroup \large}{\endgroup}        %Make eq bigger
    \newenvironment{LargeEq} {\begingroup \Large}{\endgroup}        %Make eq bigger
    \newenvironment{HugeEq} {\begingroup \Huge}{\endgroup}          %Make eq bigger!

    % =========================================
    % == GENERAL TEXT & SYMBOLS ENVIRONMENTS ==
    % =========================================
    
    % =====  TEXT  ======================
    \newcommand \Quote              {\qq}                           %Use: \Quote to use quotes
    \newcommand \Over               {\overline}                     %Use: \Bar to use just for short
    \newcommand \ForceNewLine       {$\Space$\\}                    %Use it in theorems for example
    \newcommand \ForceColumnBreak   {\vfill\null\columnbreak}       %Use only in multicols
    \newcommand \Link[2] {\underline{\texttt{\href{#1}{#2}}}}       %Use a link

    % =====  SPACES  ====================
    \DeclareMathOperator \Space     {\quad}                         %Use: \Space for a cool mega space
    \DeclareMathOperator \MegaSpace {\quad \quad}                   %Use: \MegaSpace for a cool mega mega space
    \DeclareMathOperator \MiniSpace {\;}                            %Use: \Space for a cool mini space
    
    % =====  MATH TEXT  =================
    \newcommand \Such           {\MiniSpace | \MiniSpace}           %Use: \Such like in sets
    \newcommand \Also           {\MiniSpace \text{y} \MiniSpace}    %Use: \Also so it's look cool
    \newcommand \Remember[1]    {\Space\text{\scriptsize{#1}}}      %Use: \Remember so it's look cool
    
    % =====  THEOREMS: IN SPANISH :0  ===
    \newtheorem{Theorem}        {Teorema}[section]                  %Use: \begin{Theorem}[Name]\label{Nombre}...
    \newtheorem{Corollary}      {Colorario}[Theorem]                %Use: \begin{Corollary}[Name]\label{Nombre}...
    \newtheorem{Lemma}[Theorem] {Lemma}                             %Use: \begin{Lemma}[Name]\label{Nombre}...
    \newtheorem{Definition}     {Definición}[section]               %Use: \begin{Definition}[Name]\label{Nombre}...
    \theoremstyle{break}                                            %THEOREMS START 1 SPACE AFTER Fuck!

    % =====  LOGIC  =====================
    \newcommand \lIff    {\leftrightarrow}                          %Use: \lIff for logic iff
    \newcommand \lEqual  {\MiniSpace \Leftrightarrow \MiniSpace}    %Use: \lEqual for a logic double arrow
    \newcommand \lInfire {\MiniSpace \Rightarrow \MiniSpace}        %Use: \lInfire for a logic infire
    \newcommand \lLongTo {\longrightarrow}                          %Use: \lLongTo for a long arrow
    \newcommand \lAnd    {\land}                                    %Use: \lAnd ^
    \newcommand \lOr     {\lor}                                     %Use: \lOr or symbol
    \newcommand \lNot    {\neg}                                     %Use: \lNot for negation

    % =====  FAMOUS SETS  ===============
    \DeclareMathOperator \Naturals     {\mathbb{N}}                 %Use: \Naturals por Notation
    \DeclareMathOperator \Primes       {\mathbb{P}}                 %Use: \Primes por Notation
    \DeclareMathOperator \Integers     {\mathbb{Z}}                 %Use: \Integers por Notation
    \DeclareMathOperator \Racionals    {\mathbb{Q}}                 %Use: \Racionals por Notation
    \DeclareMathOperator \Reals        {\mathbb{R}}                 %Use: \Reals por Notation
    \DeclareMathOperator \Complexs     {\mathbb{C}}                 %Use: \Complex por Notation
    \DeclareMathOperator \GenericField {\mathbb{F}}                 %Use: \GenericField por Notation
    \DeclareMathOperator \VectorSet    {\mathbb{V}}                 %Use: \VectorSet por Notation
    \DeclareMathOperator \SubVectorSet {\mathbb{W}}                 %Use: \SubVectorSet por Notation
    \DeclareMathOperator \Polynomials  {\mathbb{P}}                 %Use: \Polynomials por Notation
    \DeclareMathOperator \VectorSpace  {\VectorSet_{\GenericField}} %Use: \VectorSpace por Notation
    \DeclareMathOperator \LinealTransformation {\mathcal{T}}        %Use: \LinealTransformation for a cool T
    \DeclareMathOperator \LinTrans      {\mathcal{T}}               %Use: \LinTrans for a cool T
    \DeclareMathOperator \Laplace       {\mathcal{L}}               %Use: \LinTrans for a cool T

    % =====  CONTAINERS   ===============
    \newcommand{\Set}[1]            {\left\{ \; #1 \; \right\}}     %Use: \Set {Info} for INTELLIGENT space 
    \newcommand{\bigSet}[1]         {\big\{  \; #1 \; \big\}}       %Use: \bigSet  {Info} for space 
    \newcommand{\BigSet}[1]         {\Big\{  \; #1 \; \Big\}}       %Use: \BigSet  {Info} for space 
    \newcommand{\biggSet}[1]        {\bigg\{ \; #1 \; \bigg\}}      %Use: \biggSet {Info} for space 
    \newcommand{\BiggSet}[1]        {\Bigg\{ \; #1 \; \Bigg\}}      %Use: \BiggSet {Info} for space 
        
    \newcommand{\Wrap}[1]           {\left( #1 \right)}             %Use: \Wrap {Info} for INTELLIGENT space
    \newcommand{\bigWrap}[1]        {\big( \; #1 \; \big)}          %Use: \bigBrackets  {Info} for space 
    \newcommand{\BigWrap}[1]        {\Big( \; #1 \; \Big)}          %Use: \BigBrackets  {Info} for space 
    \newcommand{\biggWrap}[1]       {\bigg( \; #1 \; \bigg)}        %Use: \biggBrackets {Info} for space 
    \newcommand{\BiggWrap}[1]       {\Bigg( \; #1 \; \Bigg)}        %Use: \BiggBrackets {Info} for space 

    \newcommand{\Brackets}[1]       {\left[ #1 \right]}             %Use: \Brackets {Info} for INTELLIGENT space
    \newcommand{\bigBrackets}[1]    {\big[ \; #1 \; \big]}          %Use: \bigBrackets  {Info} for space 
    \newcommand{\BigBrackets}[1]    {\Big[ \; #1 \; \Big]}          %Use: \BigBrackets  {Info} for space 
    \newcommand{\biggBrackets}[1]   {\bigg[ \; #1 \; \bigg]}        %Use: \biggBrackets {Info} for space 
    \newcommand{\BiggBrackets}[1]   {\Bigg[ \; #1 \; \Bigg]}        %Use: \BiggBrackets {Info} for space 

    \newcommand{\Generate}[1]   {\left\langle #1 \right\rangle}     %Use: \Generate {Info} <>
    \newcommand{\Floor}[1]      {\left \lfloor #1 \right \rfloor}   %Use: \Floor {Info} for floor 
    \newcommand{\Ceil}[1]       {\left \lceil #1 \right \rceil }    %Use: \Ceil {Info} for ceil
    
    % =====  BETTERS MATH COMMANDS   =====
    \newcommand{\pfrac}[2]      {\Wrap{\dfrac{#1}{#2}}}             %Use: Put fractions in parentesis
    \newcommand{\Sum}           {\displaystyle \sum}                %Use: Sum to big sum
    \newcommand{\Int}           {\displaystyle \int}                %Use: Sum to big integral


    % =========================================
    % ====   LINEAL ALGEBRA & VECTORS    ======
    % =========================================

    % ===== UNIT VECTORS  ================
    \newcommand{\hati}      {\hat{\imath}}                           %Use: \hati for unit vector    
    \newcommand{\hatj}      {\hat{\jmath}}                           %Use: \hatj for unit vector    
    \newcommand{\hatk}      {\hat{k}}                                %Use: \hatk for unit vector

    % ===== MAGNITUDE  ===================
    \newcommand{\abs}[1]    {\left\lvert #1 \right\lvert}           %Use: \abs{expression} for |x|
    \newcommand{\Abs}[1]    {\left\lVert #1 \right\lVert}           %Use: \Abs{expression} for ||x||
    \newcommand{\Mag}[1]    {\left| #1 \right|}                     %Use: \Mag {Info} 
    
    \newcommand{\bVec}[1]   {\mathbf{#1}}                           %Use for bold type of vector
    \newcommand{\lVec}[1]   {\overrightarrow{#1}}                   %Use for a long arrow over a vector
    \newcommand{\uVec}[1]   {\mathbf{\hat{#1}}}                     %Use: Unitary Vector Example: $\uVec{i}

    % ===== FN LINEAL TRANSFORMATION  ====
    \newcommand{\FnLinTrans}[1]{\mathcal{T}\Wrap{#1}}               %Use: \FnLinTrans for a cool T
    \newcommand{\VecLinTrans}[1]{\mathcal{T}\pVector{#1}}           %Use: \LinTrans for a cool T
    \newcommand{\FnLinealTransformation}[1]{\mathcal{T}\Wrap{#1}}   %Use: \FnLinealTransformation

    % ===== ALL FOR DOT PRODUCT  =========
    \makeatletter                                                   %WTF! IS THIS
    \newcommand*\dotP{\mathpalette\dotP@{.5}}                       %Use: \dotP for dot product
    \newcommand*\dotP@[2] {\mathbin {                               %WTF! IS THIS            
        \vcenter{\hbox{\scalebox{#2}{$\m@th#1\bullet$}}}}           %WTF! IS THIS
    }                                                               %WTF! IS THIS
    \makeatother                                                    %WTF! IS THIS

    % === WRAPPERS FOR COLUMN VECTOR ===
    \newcommand{\pVector}[1]                                        %Use: \pVector {Matrix Notation} use parentesis
        { \ensuremath{\begin{pmatrix}#1\end{pmatrix}} }             %Example: \pVector{a\\b\\c} or \pVector{a&b&c} 
    \newcommand{\lVector}[1]                                        %Use: \lVector {Matrix Notation} use a abs 
        { \ensuremath{\begin{vmatrix}#1\end{vmatrix}} }             %Example: \lVector{a\\b\\c} or \lVector{a&b&c} 
    \newcommand{\bVector}[1]                                        %Use: \bVector {Matrix Notation} use a brackets 
        { \ensuremath{\begin{bmatrix}#1\end{bmatrix}} }             %Example: \bVector{a\\b\\c} or \bVector{a&b&c} 
    \newcommand{\Vector}[1]                                         %Use: \Vector {Matrix Notation} no parentesis
        { \ensuremath{\begin{matrix}#1\end{matrix}} }               %Example: \Vector{a\\b\\c} or \Vector{a&b&c}

    % === MAKE MATRIX BETTER  =========
    \makeatletter                                                   %Example: \begin{matrix}[cc|c]
    \renewcommand*\env@matrix[1][*\c@MaxMatrixCols c] {             %WTF! IS THIS
        \hskip -\arraycolsep                                        %WTF! IS THIS
        \let\@ifnextchar\new@ifnextchar                             %WTF! IS THIS
        \array{#1}                                                  %WTF! IS THIS
    }                                                               %WTF! IS THIS
    \makeatother                                                    %WTF! IS THIS
    
    \newcommand{\adotP}[2] {\left< #1, #2 \right> }                 %Use for <x, y>
    \newcommand{\wdotP}[2] {\Wrap{ #1, #2 } }                       %Use for (x, y)
    \newcommand{\cdotP}[2] {\Wrap{ #1 \dotP #2 } }                  %Use for (x * y)


    % =========================================
    % =======   FAMOUS FUNCTIONS   ============
    % =========================================

    % == TRIGONOMETRIC FUNCTIONS  ====
    \newcommand{\Cos}[1] {\cos\Wrap{#1}}                            %Simple wrappers
    \newcommand{\Sin}[1] {\sin\Wrap{#1}}                            %Simple wrappers
    \newcommand{\Tan}[1] {tan\Wrap{#1}}                             %Simple wrappers
    
    \newcommand{\Sec}[1] {sec\Wrap{#1}}                             %Simple wrappers
    \newcommand{\Csc}[1] {csc\Wrap{#1}}                             %Simple wrappers
    \newcommand{\Cot}[1] {cot\Wrap{#1}}                             %Simple wrappers

    % === COMPLEX ANALYSIS TRIG ======
    \newcommand \Cis[1]  {\Cos{#1} + i \Sin{#1}}                    %Use: \Cis for cos(x) + i sin(x)
    \newcommand \pCis[1] {\Wrap{\Cis{#1}}}                          %Use: \pCis for the same with parantesis
    \newcommand \bCis[1] {\Brackets{\Cis{#1}}}                      %Use: \bCis for the same with Brackets


    % =========================================
    % ===========     CALCULUS     ============
    % =========================================

    % ====== TRANSFORMS =============
    \newcommand{\FourierT}[1]   {\mathscr{F} \left\{ #1 \right\} }  %Use: \FourierT {Funtion}
    \newcommand{\InvFourierT}[1]{\mathscr{F}^{-1}\left\{#1\right\}} %Use: \InvFourierT {Funtion}

    % ====== DERIVATIVES ============
    \newcommand \MiniDerivate[1][x]   {\dfrac{d}{d #1}}             %Use: \MiniDerivate[var] for simple use [var]
    \newcommand \Derivate[2]          {\dfrac{d \; #1}{d #2}}       %Use: \Derivate [f(x)][x]
    \newcommand \MiniUpperDerivate[2] {\dfrac{d^{#2}}{d#1^{#2}}}    %Mini Derivate High Orden Derivate -- [x][pow]
    \newcommand \UpperDerivate[3] {\dfrac{d^{#3} \; #1}{d#2^{#3}}}  %Complete High Orden Derivate -- [f(x)][x][pow]
    
    \newcommand \MiniPartial[1][x] {\dfrac{\partial}{\partial #1}}  %Use: \MiniDerivate for simple use [var]
    \newcommand \Partial[2] {\dfrac{\partial \; #1}{\partial #2}}   %Complete Partial Derivate -- [f(x)][x]
    \newcommand \MiniUpperPartial[2]                                %Mini Derivate High Orden Derivate -- [x][pow] 
        {\dfrac{\partial^{#2}}{\partial #1^{#2}}}                   %Mini Derivate High Orden Derivate
    \newcommand \UpperPartial[3]                                    %Complete High Orden Derivate -- [f(x)][x][pow]
        {\dfrac{\partial^{#3} \; #1}{\partial#2^{#3}}}              %Use: \UpperDerivate for simple use

    \DeclareMathOperator \Evaluate  {\Big|}                         %Use: \Evaluate por Notation

    % ====== INTEGRALS ============
    \newcommand{\inftyInt} {\int_{-\infty}^{\infty}}                %Use: \inftyInt for simple integrants
    
        
% =======================================================
% ===========      COLOR: MATERIAL DESIGN     ===========
% =======================================================

    % =====  COLORS ==================
    \definecolor{RedMD}{HTML}{F44336}                               %Use: Color :D        
    \definecolor{Red100MD}{HTML}{FFCDD2}                            %Use: Color :D        
    \definecolor{Red200MD}{HTML}{EF9A9A}                            %Use: Color :D        
    \definecolor{Red300MD}{HTML}{E57373}                            %Use: Color :D        
    \definecolor{Red700MD}{HTML}{D32F2F}                            %Use: Color :D 

    \definecolor{PurpleMD}{HTML}{9C27B0}                            %Use: Color :D        
    \definecolor{Purple100MD}{HTML}{E1BEE7}                         %Use: Color :D        
    \definecolor{Purple200MD}{HTML}{EF9A9A}                         %Use: Color :D        
    \definecolor{Purple300MD}{HTML}{BA68C8}                         %Use: Color :D        
    \definecolor{Purple700MD}{HTML}{7B1FA2}                         %Use: Color :D 

    \definecolor{IndigoMD}{HTML}{3F51B5}                            %Use: Color :D        
    \definecolor{Indigo100MD}{HTML}{C5CAE9}                         %Use: Color :D        
    \definecolor{Indigo200MD}{HTML}{9FA8DA}                         %Use: Color :D        
    \definecolor{Indigo300MD}{HTML}{7986CB}                         %Use: Color :D        
    \definecolor{Indigo700MD}{HTML}{303F9F}                         %Use: Color :D 

    \definecolor{BlueMD}{HTML}{2196F3}                              %Use: Color :D        
    \definecolor{Blue100MD}{HTML}{BBDEFB}                           %Use: Color :D        
    \definecolor{Blue200MD}{HTML}{90CAF9}                           %Use: Color :D        
    \definecolor{Blue300MD}{HTML}{64B5F6}                           %Use: Color :D        
    \definecolor{Blue700MD}{HTML}{1976D2}                           %Use: Color :D        
    \definecolor{Blue900MD}{HTML}{0D47A1}                           %Use: Color :D  

    \definecolor{CyanMD}{HTML}{00BCD4}                              %Use: Color :D        
    \definecolor{Cyan100MD}{HTML}{B2EBF2}                           %Use: Color :D        
    \definecolor{Cyan200MD}{HTML}{80DEEA}                           %Use: Color :D        
    \definecolor{Cyan300MD}{HTML}{4DD0E1}                           %Use: Color :D        
    \definecolor{Cyan700MD}{HTML}{0097A7}                           %Use: Color :D        
    \definecolor{Cyan900MD}{HTML}{006064}                           %Use: Color :D 

    \definecolor{TealMD}{HTML}{009688}                              %Use: Color :D        
    \definecolor{Teal100MD}{HTML}{B2DFDB}                           %Use: Color :D        
    \definecolor{Teal200MD}{HTML}{80CBC4}                           %Use: Color :D        
    \definecolor{Teal300MD}{HTML}{4DB6AC}                           %Use: Color :D        
    \definecolor{Teal700MD}{HTML}{00796B}                           %Use: Color :D        
    \definecolor{Teal900MD}{HTML}{004D40}                           %Use: Color :D 

    \definecolor{GreenMD}{HTML}{4CAF50}                             %Use: Color :D        
    \definecolor{Green100MD}{HTML}{C8E6C9}                          %Use: Color :D        
    \definecolor{Green200MD}{HTML}{A5D6A7}                          %Use: Color :D        
    \definecolor{Green300MD}{HTML}{81C784}                          %Use: Color :D        
    \definecolor{Green700MD}{HTML}{388E3C}                          %Use: Color :D        
    \definecolor{Green900MD}{HTML}{1B5E20}                          %Use: Color :D

    \definecolor{AmberMD}{HTML}{FFC107}                             %Use: Color :D        
    \definecolor{Amber100MD}{HTML}{FFECB3}                          %Use: Color :D        
    \definecolor{Amber200MD}{HTML}{FFE082}                          %Use: Color :D        
    \definecolor{Amber300MD}{HTML}{FFD54F}                          %Use: Color :D        
    \definecolor{Amber700MD}{HTML}{FFA000}                          %Use: Color :D        
    \definecolor{Amber900MD}{HTML}{FF6F00}                          %Use: Color :D

    \definecolor{OrangeMD}{HTML}{ff9800}                            %Use: Color :D        
    \definecolor{Orange100MD}{HTML}{ffe0b2}                         %Use: Color :D        
    \definecolor{Orange200MD}{HTML}{ffcc80}                         %Use: Color :D        
    \definecolor{Orange300MD}{HTML}{ffb74d}                         %Use: Color :D        
    \definecolor{Orange700MD}{HTML}{fb8c00}                         %Use: Color :D        
    \definecolor{Orange900MD}{HTML}{ef6c00}                         %Use: Color :D

    \definecolor{BlueGreyMD}{HTML}{607D8B}                          %Use: Color :D        
    \definecolor{BlueGrey100MD}{HTML}{CFD8DC}                       %Use: Color :D        
    \definecolor{BlueGrey200MD}{HTML}{B0BEC5}                       %Use: Color :D        
    \definecolor{BlueGrey300MD}{HTML}{90A4AE}                       %Use: Color :D        
    \definecolor{BlueGrey700MD}{HTML}{455A64}                       %Use: Color :D        
    \definecolor{BlueGrey900MD}{HTML}{263238}                       %Use: Color :D        

    \definecolor{DeepPurpleMD}{HTML}{673AB7}                        %Use: Color :D

    \definecolor{SolarizedBase}{HTML}{fdf6e3}                       %Use: Color :D
    \definecolor{SolarizedFont}{HTML}{073642}                       %Use: Color :D

    % =====  ENVIRONMENT ==============
    \newcommand{\Color}[2]{\textcolor{#1}{#2}}                      %Simple color environment
    \newenvironment{ColorText}[1]                                   %Use: \begin{ColorText}
        { \leavevmode\color{#1}\ignorespaces }                      %That's is!


% =======================================================
% ===========           CODE EDITING          ===========
% =======================================================

    \newcommand{\fontCode}        { \ttfamily\bfseries }            %Use: \fontCode for font
    \newcommand{\fontCodeTiny}    { \fontCode\tiny }                %Sizes
    \newcommand{\fontCodeFoot}    { \fontCode\footnotesize }        %Sizes
    \newcommand{\fontCodeScript}  { \fontCode\scriptsize }          %Sizes
    \newcommand{\fontCodeCostume} { \fontCode\fontsize{10}{7} }     %Sizes
   

    % =====  CODE EDITOR =============
    \lstdefinestyle{CompilandoStyle} {                              %This is Code Style
        backgroundcolor     = \color{BlueGrey900MD},                %Background Color  
        basicstyle          = \fontCodeTiny\color{white},           %Style of text
        commentstyle        = \color{BlueGrey200MD},                %Comment style
        stringstyle         = \color{Green300MD},                   %String style
        keywordstyle        = \color{Blue300MD},                    %keywords style
        numberstyle         = \tiny\color{TealMD},                  %Size of a number
        frame               = none,                                 %Adds a frame around the code
        breakatwhitespace   = true,                                 %Style   
        breaklines          = true,                                 %Style   
        showstringspaces    = false,                                %Hate those spaces                  
        breaklines          = true,                                 %Style                   
        keepspaces          = true,                                 %Style                   
        numbers             = left,                                 %Style                   
        numbersep           = 10pt,                                 %Style 
        xleftmargin         = \parindent,                           %Style 
        tabsize             = 4,                                    %Style
        inputencoding       = utf8/latin1                           %Allow me to use special chars
    }

    % =====  CODE EDITOR =============
    \lstdefinestyle{CompilandoStylePurity} {                        %This is Code Style
        backgroundcolor     = \color{white},                        %Background Color  
        basicstyle          = \fontCodeTiny\color{BlueGrey900MD},   %Style of text
        commentstyle        = \color{Green300MD},                   %Comment style
        stringstyle         = \color{Teal700MD},                    %String style
        keywordstyle        = \color{Blue700MD},                    %keywords style
        numberstyle         = \tiny\color{TealMD},                  %Size of a number
        frame               = none,                                 %Adds a frame around the code
        breakatwhitespace   = true,                                 %Style   
        breaklines          = true,                                 %Style   
        showstringspaces    = false,                                %Hate those spaces                  
        breaklines          = true,                                 %Style                   
        keepspaces          = true,                                 %Style                   
        numbers             = left,                                 %Style                   
        numbersep           = 11pt,                                 %Style 
        xleftmargin         = \parindent,                           %Style 
        tabsize             = 4,                                    %Style
        inputencoding       = utf8/latin1                           %Allow me to use special chars
    }

    % =====  CODE EDITOR =============
    \lstdefinestyle{CompilandoStyleSolarized} {                     %This is Code Style
        backgroundcolor     = \color{SolarizedBase},                %Background Color  
        basicstyle          = \fontCodeFoot\color{SolarizedFont},   %Style of text
        commentstyle        = \color{Green300MD},                   %Comment style
        stringstyle         = \color{Teal700MD},                    %String style
        keywordstyle        = \color{Blue700MD},                    %keywords style
        numberstyle         = \tiny\color{TealMD},                  %Size of a number
        frame               = none,                                 %Adds a frame around the code
        breakatwhitespace   = true,                                 %Style   
        breaklines          = true,                                 %Style   
        showstringspaces    = false,                                %Hate those spaces                  
        breaklines          = true,                                 %Style                   
        keepspaces          = true,                                 %Style                   
        numbers             = none,                                 %Style                   
        tabsize             = 4,                                    %Style
        inputencoding       = utf8/latin1                           %Allow me to use special chars
    }
 
    \lstset{style = CompilandoStyleSolarized}                          %Use this style

\begin{document}

\title{Tarea 1: Gráficas y Juegos}

\author{
    Oscar Andrés Rosas Hernández [SoyOscarRH@gmail.com / 417024956]                 \\
    Rodrigo Alfredo Lemus Palma  [rodrigolemus97@ciencias.unam.mx / 417006954]      \\[1em]
    \textit{
    Facultad de Ciencias, UNAM, CDMX, México
    }
}

\markboth{Ejercicios}{Graficas}

\maketitle
    \section*{Primer Problema}
        Observa la figura 1 y con base en tu criterio plantea una gráfica que la represente, explica 
        el porqué de tu planteamiento.
        \begin{figure}[h]
            \includegraphics[width=0.6\textwidth]{1}
        \end{figure}

    \section*{Segundo Problema}
        Cuatro estudiantes universitarios Fernando (F), Lourdes (L), Mateo (M) y Pedro (P)
        están viendo un partido de fútbol en la televisión en un bar. 
        
        Durante el entretiempo, entablan una discusión sobre qué equipos de fútbol 
        que han visto jugar en persona: a los New England Patriots (NE), 
        New York Giants (NG), Dallas Cowboys (DC) y Chicago Bears (CB). 
        
        Esto es lo que dicen:
        \begin{itemize}
            \item \textbf{F}: NE, NG, CB
            \item \textbf{L}: NE, DC, CB
            \item \textbf{M}: NE, NG, DC
            \item \textbf{P}: NG, DC, CB
        \end{itemize}
        
        Propón una gráfica que represente esta situación. \\

        \begin{SmallIndentation}[1em]
            \textbf{Solución}:

            Lo que se nos ocurrio fue usar no solo una gráfica $G$, sino una bipartita,
            de tal manera que $U$ representen a las personas y $V$ a los equipos
            mientras que las aristas, que una personas vio a un equipo en vivo.

            Siendo mas exacto tenemos que:
            \begin{itemize}
                \item $Vertices(U) = \Set{F, L, M, P}$
                \item $Vertices(V) = \Set{NE, NG, DC, CB}$
                \item $Aristas(G) = \Set{
                    \mathbf{F}NE, \mathbf{F}NG, \mathbf{F}CB,
                    \mathbf{L}NE, \mathbf{L}DC, \mathbf{L}CB, 
                    \mathbf{M}NE, \mathbf{M}NG, \mathbf{M}DC,
                    \mathbf{P}NG, \mathbf{P}DC, \mathbf{P}CB
                }$
            \end{itemize}

            \begin{figure}[h]
                \includegraphics[width=0.4\textwidth]{2}
            \end{figure}
        \end{SmallIndentation}
    
    \section*{Tercer Problema}
        Construye una gráfica G, tal que $|V(G)| = 12$, $|A(G)| = 33$, 
        seis vértices tengan grado 4 y seis vértices de grado 7.

        \begin{itemize}
            \item En azul los que tienen grado 4
            \item En verde los que tienen grado 7
        \end{itemize}

        \begin{figure}[h]
            \includegraphics[width=0.4\textwidth]{3}
        \end{figure}
        
    \section*{Cuarto Problema}

        ¿Cuál es el tamaño máximo posible para las siguientes gráficas?
        \begin{itemize}
            \item De orden 3.
            \item De orden 4.
            \item De orden 5.
            \item De orden n con $n \in N$.
        \end{itemize}

        \begin{figure}[h]
            \includegraphics[width=0.5\textwidth]{41}
        \end{figure}

        \begin{figure}[h]
            \includegraphics[width=0.5\textwidth]{42}
        \end{figure}

    \clearpage
    \begin{landscape}
        \section*{Quinto Problema}

        Sea $S = \Set{2, 3, 4, 7, 11, 13}$. Dibuja la gráfica simple $G$ cuyo 
        conjunto de vértices es $S$ y tal que $ij \in A(G)$ para $i, j \in S$ si 
        $i + j \in S$ o $|i - j| \in S$.
        Hagamos una matriz de adjacencia:

        \begin{SmallIndentation}[1em]
            \textbf{Solución}:

            Hagamos los fragmentos de la matriz:

            \begin{tabular}{|l|l|l|l|l|l|l|}
                \hline
                    & 2 & 3                         & 4                         & 7                         & 11                           & 13 \\ \hline
                2   & @ & 0 $(2+3=5$ y $|2-3|=1)$   & 1 $(2+4=6$ y $|2-4|=2)$   & 0 $(2+7=9$ y $|2-7|=5)$   & 1 $(2+11=13$ y $|2-11|=9)$   & 1 $(2+13=15$ y $|2-13|=11)$ \\ \hline
                3   & @ & @                         & 1 $(3+4=7$ y $|3-4|=1)$   & 0 $(3+7=10$ y $|3-7|=4)$  & 0 $(3+11=14$ y $|3-11|=8)$   & 0 $(3+13=16$ y $|3-13|=10)$ \\ \hline
                4   & @ & @                         & @                         & 1 $(4+7=11$ y $|3-7|=4)$  & 1 $(4+11=15$ y $|4-11|=7)$   & 0 $(4+13=17$ y $|4-13|=9)$  \\ \hline
                7   & @ & @                         & @                         & @                         & 1 $(7+11=18$ y $|7-11|=4)$   & 0 $(7+13=20$ y $|7-13|=6)$  \\ \hline
                11  & @ & @                         & @                         & @                         & @                            & 1 $(11+13=24$ y $|11-13|=2)$\\ \hline
                13  & @ & @ & @ & @ & @ & @ \\ \hline
            \end{tabular}
            

            Bueno, ahora podemos llenarla con todos los valores:

            \begin{tabular}{|l|l|l|l|l|l|l|}
                \hline
                    & 2 & 3 & 4 & 7 & 11 & 13   \\ \hline
                2   & 0 & 0 & 1 & 0 & 1  & 1    \\ \hline
                3   & 0 & 0 & 1 & 1 & 0  & 0    \\ \hline
                4   & 1 & 1 & 0 & 1 & 1  & 0    \\ \hline
                7   & 0 & 0 & 1 & 0 & 1  & 0    \\ \hline
                11  & 1 & 0 & 1 & 1 & 0  & 1    \\ \hline
                13  & 1 & 0 & 0 & 0 & 1  & 0    \\ \hline
            \end{tabular}

            \begin{figure}[h]
                \includegraphics[width=0.5\textwidth]{5}
            \end{figure}
        \end{SmallIndentation}
    \end{landscape}

    \section*{Sexto Problema}
        Crea tu propio conjunto de enteros S con $|S| > 4$ y dibuja 
        la gráfica simple G cuyo conjunto de vértices es S y tal que 
        $ij \in A(G)$ si i y j están relacionados por alguna 
        regla impuesta a j e i.

        \begin{SmallIndentation}[1em]
            \textbf{Solución}:

            Digamos que $S = {1, 2, 3, 4, 5}$
            Y digamos que $ij \in A(G)$ $i, j \in S$ y $i + j \in A(G)$.

            \begin{figure}[h]
                \includegraphics[width=0.5\textwidth]{6}
            \end{figure}

        \end{SmallIndentation}

    \section*{Septimo Problema}

        Determina el orden, tamaño y el grado de cada vértice de la gráfica de la figura 2.

        \begin{figure}[h]
            \includegraphics[width=0.8\textwidth]{7}
        \end{figure}

    \clearpage
    \section*{Octavo Problema}
        La figura 3 representa las moléculas químicas de metano ($CH_4$) y propano ($C_3H_8$).

        \begin{itemize}
            \item 
                a) Considerando estos diagramas como 
                gráficas, ¿qué puedes decir acerca de los vértices 
                que representan átomos de carbono (C) y átomos de hidrógeno (H)?
                \begin{itemize}
                    \item Los carbonos tiene todos grado 4.
                    \item Los hidrogenos tienen grado 1 (por lo tanto son vertices terminales).
                \end{itemize}

                \begin{figure}[h]
                    \includegraphics[width=0.5\textwidth]{8}
                \end{figure}

            \item 
                b) La fórmula $C_5H_{12}$ representa a diferentes moléculas orgánicas, 
                sabiendo que el carbono es tetravalente, dibuja las diferentes 
                gráficas de sus isómeros.

                \begin{figure}[h]
                    \includegraphics[width=0.3\textwidth]{81}
                    \includegraphics[width=0.3\textwidth]{82}
                    \includegraphics[width=0.3\textwidth]{83}
                \end{figure}

            \item 
                c) Hay dos moléculas químicas diferentes con la fórmula $C_4H_{10}$, dibuja las diferentes 
                gráficas de las móleculas.

                \begin{figure}[h]
                    \includegraphics[width=0.2\textwidth]{84}
                    \includegraphics[width=0.2\textwidth]{85}
                \end{figure}
        \end{itemize}

    \clearpage

    \section*{Noveno Problema}
        Demuestra que para cualquier gráfica G con 
        $|V(G)| \geq 2$, existen $a, b \in V (G)$, tal que $grado_G(a) = grado_G(b)$.

        % ======== DEMOSTRACION ========
        \begin{SmallIndentation}[1em]
            \textbf{Demostración}:

            Va, vamos a demostrarlo por contradicción.
            
            Recordemos que hay un límite al grado de los vertices, en el caso
            de una grafica simple es $n - 1$. Entonces supongamos que todos son diferentes, entonces 
            creamos una sucesión de dichos grados de los vertices.
            \begin{itemize}
                \item  $grado(u_1) = 0$
                \item  $grado(u_2) = 1$
                \item  $\dots$
                \item  $grado(u_n) = n - 1$
            \end{itemize}

            Ve que como estamos generando $n$ naturales (considerando el cero) que tiene como tope maximo $n -1$
            la unica secuencia que no tiene repetidos es $0, 1, \dots, n-1$.

            Entonces en una grafica con $n$ vertices tenemos que el vertices con grado 
            mayor es $n - 1$, pero piensa que significa que un vertice tenga $n - 1$ aristas, pues
            que esta conectado a todos los demas.

            Pero eso no puede ser porque dijimos que $grado(u_1) = 0$, es decir no hay una grafica donde un nodo
            este conectado a todos los demas y al mismo tiempo haya un vertice que no tenga aristas.

            Contradicción.

            QED.

        \end{SmallIndentation}

    \clearpage
    \section*{Decimo Problema}
        Determina todos las subgráficas inducidas de la gráfica G.

        \begin{figure}[h]
            \includegraphics[width=0.2\textwidth]{101}
            \includegraphics[width=0.5\textwidth]{102}
            \includegraphics[width=0.4\textwidth]{103}
        \end{figure}

    \clearpage
    \section*{Decimo Primer Problema}

        Escribe las matrices de incidencia y adyacencia de la siguiente gráfica.
        \begin{figure}[h]
            \includegraphics[width=0.5\textwidth]{11}
        \end{figure}

        Matriz de adjacencia:

        \begin{tabular}{|l|l|l|l|l|l|}
            \hline
                  & $x_1$ & $x_2$ & $x_3$ & $x_4$ & $x_5$ \\ \hline
            $x_1$ & 0     & 1     & 0     & 0     & 1     \\ \hline
            $x_2$ & 1     & 0     & 1     & 1     & 1     \\ \hline
            $x_3$ & 0     & 1     & 0     & 1     & 0     \\ \hline
            $x_4$ & 0     & 1     & 1     & 0     & 1     \\ \hline
            $x_5$ & 1     & 1     & 0     & 1     & 0     \\ \hline
        \end{tabular}
        \\[3em]

        Matriz de incidencia:

        \begin{tabular}{|l|l|l|l|l|l|}
            \hline
                  & $x_1$ & $x_2$ & $x_3$ & $x_4$ & $x_5$ \\ \hline
            $a_1$ & 1     & 1     & 0     & 0     & 0     \\ \hline
            $a_2$ & 0     & 1     & 1     & 0     & 0     \\ \hline
            $a_3$ & 0     & 0     & 1     & 1     & 0     \\ \hline
            $a_4$ & 0     & 0     & 0     & 1     & 1     \\ \hline
            $a_5$ & 1     & 0     & 0     & 0     & 1     \\ \hline
            $a_6$ & 0     & 1     & 0     & 0     & 1     \\ \hline
            $a_7$ & 0     & 1     & 0     & 1     & 0     \\ \hline
        \end{tabular}

    \clearpage
    \section*{Decimo Segundo Problema}
        Cierta aerolínea tiene vuelos hacia distintas ciudades. 
        Ocho de estas ciudades se indican con $C_1, C_2, \dots , C_8$.
        Esta aerolínea tiene vuelos directos entre ciertos pares de ciudades. 

        Esta información se da en la siguiente matriz:

        \begin{equation*}
            \pVector {
                0 1 0 0 1 1 0 0 \\ 
                1 0 0 0 0 0 1 0 \\
                0 0 0 1 0 1 0 1 \\
                0 0 1 0 0 0 0 1 \\
                1 0 0 0 0 1 1 0 \\
                1 0 1 0 1 0 0 0 \\
                0 1 0 0 1 0 0 0 \\
                0 0 1 1 0 0 0 0 \\
            }
        \end{equation*}

        donde $a_{ij} = 1$ significa que hay un vuelo directo entre las ciudades $C_i y C_j$. 
        Dibuja la gráfica que representa los vuelos entre las ocho ciudades.

        \begin{figure}[h]
            \includegraphics[width=0.5\textwidth]{12}
        \end{figure}

    \clearpage
    \section*{Decimo Tercero Problema}
        Sea A la matriz de adyacencia para $K_3$, la gráfica completa de tres vértices. 
        Utiliza inducción matemática para demostrar que para cada entero positivo n, 
        todas las entradas a lo largo de la diagonal principal de $A_n$ son iguales 
        entre si y todas las entradas que no se encuentran en la diagonal principal
        son iguales entre si.

        % ======== DEMOSTRACION ========
        \begin{SmallIndentation}[1em]
            \textbf{Demostración}:

            Va, para $K_3$ hagamos su matriz de adjacencia:

            \begin{tabular}{|l|l|l|l|}
                \hline
                      & $x_1$ & $x_2$ & $x_3$ \\ \hline
                $x_1$ & 0     & 1     & 1     \\ \hline
                $x_2$ & 1     & 0     & 1     \\ \hline
                $x_3$ & 1     & 1     & 0     \\ \hline
            \end{tabular}

            Ahora, para este el caso base se cumple, se cumple que toda la diagonal principal de $A_n$ es igual, todos
            los valores de esta son cero, y tambien pasa que todos los otros valores son iguales, son 1.

            Ahora supongamos que esta proposición se cumple hasta $k$.

            Ahora, veamos que es lo que pasa cuando intentamos añadir ese vértice $k + 1$.

            Bueno, toda la matriz de adjacencia va a seguir igual, solo se van a agregar una ultima fila
            y una ultima columna, la correspondiente a $x_{k+1}$, ahora, veamos que pasa en
            $A_{k+1}[k + 1][k+1]$ (es decir la última casilla, la última casilla de la diagonal principal),
            esta puede tener dos valores para una grafica simple, 0 o 1, ahora no puede ser 1, porque eso
            significaría que hay una arista entre dos vertices iguales (el vertice $x_{k+1}$), y eso 
            no es posible en una grafica simple, por lo tanto tiene que ser cero, y al ser cero
            eso quiere decir que se sigue cumpliendo que todos los elementos de la diagonal principal son iguales,
            todos son cero.

            Ahora veamos que es lo que pasa en las otras casillas de la nueva fila y columna, aquí por definición
            el valor tiene que ser uno, porque un uno en una casilla $A_{k+1}[k + 1][i]$ o $A_{k+1}[i][k + 1]$,
            quiere decir que hay una arista entre $x_{k+1}$ y $x_{i}$ (con $i \neq k + 1$), y eso por definición
            de las $A_n$ se cumpla, por eso son graficas completas, por lo tanto se cumple que las nuevas casillas
            que no son la que esta en la diagonal principal son iguales, todas son 1.

            QED.

        \end{SmallIndentation}

    \clearpage
    \section*{Decimo Cuarto Problema}

        Al etiquetar adecuadamente los vértices, demuestra 
        que las dos gráficas de la figura 6 son isomorfas.
        \begin{figure}[h]
            \includegraphics[width=0.5\textwidth]{141}
        \end{figure}

        Explica por qué las dos gráficas de la 
        figura 7 no son isomorfas.
        \begin{figure}[h]
            \includegraphics[width=0.5\textwidth]{142}
        \end{figure}

        Aqui hay dos ideas equivalentes que decir:
        \begin{itemize}
            \item Por un lado como vimos en clase los isomorfismos
            preservan caminos, y podemos ver claramente que hay un
            camino en la segunda grafica entre los vertices de grafo 3.

            Pero no existe dicho camino en la figura 1, por lo tanto no 
            son isomorfas.

            \item Otra forma de decirlo es que en la segunda figura los
            4 vertices de grado 3 forman una subgrafica conexa, pero 
            en la primera figura no es asi, por lo tanto tampoco pueden ser
            un isomorfismo.
        \end{itemize}




\end{document}