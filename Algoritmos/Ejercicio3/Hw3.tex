% ****************************************************************************************
% *****************          EJERCICIO DE ALGORITMOS           ***************************
% ****************************************************************************************


% =======================================================
% =======         HEADER FOR DOCUMENT        ============
% =======================================================
    
    % *********   HEADERS AND FOOTERS ********
    \def\ProjectAuthorLink{https://github.com/SoyOscarRH}           %Just to keep it in line
    \def\ProjectNameLink{\ProjectAuthorLink/Proyect}                %Link to Proyect

    % *********   DOCUMENT ITSELF   **************
    \documentclass[12pt, fleqn]{article}                            %Type of document and size of font
    \usepackage[spanish]{babel}                                     %Please use spanish
    \usepackage[utf8]{inputenc}                                     %Please use spanish - UFT
    \usepackage[margin = 1.2in]{geometry}                           %Margins and Geometry pacakge
    \usepackage{ifthen}                                             %Allow simple programming
    \usepackage{hyperref}                                           %Create MetaData for a PDF and LINKS!
    \usepackage{pdfpages}                                           %Create MetaData for a PDF and LINKS!
    \hypersetup{pageanchor = false}                                 %Solve 'double page 1' warnings in build
    \setlength{\parindent}{0pt}                                     %Eliminate ugly indentation
    \author{Oscar Andrés Rosas}                                     %Who I am

    % *********   LANGUAJE    *****************
    \usepackage[T1]{fontenc}                                        %Please use spanish
    \usepackage{textcmds}                                           %Allow us to use quoutes
    \usepackage{changepage}                                         %Allow us to use identate paragraphs
    \usepackage{anyfontsize}                                        %All the sizes

    % *********   MATH AND HIS STYLE  *********
    \usepackage{ntheorem, amsmath, amssymb, amsfonts}               %All fucking math, I want all!
    \usepackage{mathrsfs, mathtools, empheq}                        %All fucking math, I want all!
    \usepackage{cancel}                                             %Negate symbol
    \usepackage{centernot}                                          %Allow me to negate a symbol
    \decimalpoint                                                   %Use decimal point

    % *********   GRAPHICS AND IMAGES *********
    \usepackage{graphicx}                                           %Allow to create graphics
    \usepackage{float}                                              %For images
    \usepackage{wrapfig}                                            %Allow to create images
    \graphicspath{ {Graphics/} }                                    %Where are the images :D

    % *********   LISTS AND TABLES ***********
    \usepackage{listings, listingsutf8}                             %We will be using code here
    \usepackage[inline]{enumitem}                                   %We will need to enumarate
    \usepackage{tasks}                                              %Horizontal lists
    \usepackage{longtable}                                          %Lets make tables awesome
    \usepackage{booktabs}                                           %Lets make tables awesome
    \usepackage{tabularx}                                           %Lets make tables awesome
    \usepackage{multirow}                                           %Lets make tables awesome
    \usepackage{multicol}                                           %Create multicolumns

    % *********   HEADERS AND FOOTERS ********
    \usepackage{fancyhdr}                                           %Lets make awesome headers/footers
    \pagestyle{fancy}                                               %Lets make awesome headers/footers
    \setlength{\headheight}{16pt}                                   %Top line
    \setlength{\parskip}{0.5em}                                     %Top line
    \renewcommand{\footrulewidth}{0.5pt}                            %Bottom line

    \lhead {                                                        %Left Header
        \hyperlink{section.\arabic{section}}                        %Make a link to the current chapter
        {\normalsize{\textsc{\nouppercase{\leftmark}}}}             %And fot it put the name
    }

    \rhead {                                                        %Right Header
        \hyperlink{section.\arabic{section}.\arabic{subsection}}    %Make a link to the current chapter
            {\footnotesize{\textsc{\nouppercase{\rightmark}}}}      %And fot it put the name
    }

    \rfoot{\textsc{\small{\hyperref[sec:Index]{Ve al Índice}}}}     %This will always be a footer  

    \fancyfoot[L]{                                                  %Algoritm for a changing footer
        \ifthenelse{\isodd{\value{page}}}                           %IF ODD PAGE:
            {\href{https://compilandoconocimiento.com/nosotros/}    %DO THIS:
                {\footnotesize                                      %Send the page
                    {\textsc{Oscar Andrés Rosas}}}}                 %Send the page
            {\href{https://compilandoconocimiento.com}              %ELSE DO THIS: 
                {\footnotesize                                      %Send the author
                    {\textsc{Compilando Conocimiento}}}}            %Send the author
    }
    
    
    
% =======================================================
% ===================   COMMANDS    =====================
% =======================================================

    % =========================================
    % =======   NEW ENVIRONMENTS   ============
    % =========================================
    \newenvironment{Indentation}[1][0.75em]                         %Use: \begin{Inde...}[Num]...\end{Inde...}
        {\begin{adjustwidth}{#1}{}}                                 %If you dont put nothing i will use 0.75 em
        {\end{adjustwidth}}                                         %This indentate a paragraph
    \newenvironment{SmallIndentation}[1][0.75em]                    %Use: The same that we upper one, just 
        {\begin{adjustwidth}{#1}{}\begin{footnotesize}}             %footnotesize size of letter by default
        {\end{footnotesize}\end{adjustwidth}}                       %that's it

    \newenvironment{MultiLineEquation}[1]                           %Use: To create MultiLine equations
        {\begin{equation}\begin{alignedat}{#1}}                     %Use: \begin{Multi..}{Num. de Columnas}
        {\end{alignedat}\end{equation}}                             %And.. that's it!
    \newenvironment{MultiLineEquation*}[1]                          %Use: To create MultiLine equations
        {\begin{equation*}\begin{alignedat}{#1}}                    %Use: \begin{Multi..}{Num. de Columnas}
        {\end{alignedat}\end{equation*}}                            %And.. that's it!
    

    % =========================================
    % == GENERAL TEXT & SYMBOLS ENVIRONMENTS ==
    % =========================================
    
    % =====  TEXT  ======================
    \newcommand \Quote {\qq}                                        %Use: \Quote to use quotes
    \newcommand \Over {\overline}                                   %Use: \Bar to use just for short
    \newcommand \ForceNewLine {$\Space$\\}                          %Use it in theorems for example

    % =====  SPACES  ====================
    \DeclareMathOperator \Space {\quad}                             %Use: \Space for a cool mega space
    \DeclareMathOperator \MegaSpace {\quad \quad}                   %Use: \MegaSpace for a cool mega mega space
    \DeclareMathOperator \MiniSpace {\;}                            %Use: \Space for a cool mini space
    
    % =====  MATH TEXT  =================
    \newcommand \Such {\MiniSpace | \MiniSpace}                     %Use: \Such like in sets
    \newcommand \Also {\MiniSpace \text{y} \MiniSpace}              %Use: \Also so it's look cool
    \newcommand \Remember[1]{\Space\text{\scriptsize{#1}}}          %Use: \Remember so it's look cool
    
    % =====  THEOREMS  ==================
    \newtheorem{Theorem}{Teorema}[section]                          %Use: \begin{Theorem}[Name]\label{Nombre}...
    \newtheorem{Corollary}{Colorario}[Theorem]                      %Use: \begin{Corollary}[Name]\label{Nombre}...
    \newtheorem{Lemma}[Theorem]{Lemma}                              %Use: \begin{Lemma}[Name]\label{Nombre}...
    \newtheorem{Definition}{Definición}[section]                    %Use: \begin{Definition}[Name]\label{Nombre}...
    \theoremstyle{break}                                            %THEOREMS START 1 SPACE AFTER

    % =====  LOGIC  =====================
    \newcommand \lIff    {\leftrightarrow}                          %Use: \lIff for logic iff
    \newcommand \lEqual  {\MiniSpace \Leftrightarrow \MiniSpace}    %Use: \lEqual for a logic double arrow
    \newcommand \lInfire {\MiniSpace \Rightarrow \MiniSpace}        %Use: \lInfire for a logic infire
    \newcommand \lLongTo {\longrightarrow}                          %Use: \lLongTo for a long arrow

    % =====  FAMOUS SETS  ===============
    \DeclareMathOperator \Naturals     {\mathbb{N}}                 %Use: \Naturals por Notation
    \DeclareMathOperator \Primes       {\mathbb{P}}                 %Use: \Primes por Notation
    \DeclareMathOperator \Integers     {\mathbb{Z}}                 %Use: \Integers por Notation
    \DeclareMathOperator \Racionals    {\mathbb{Q}}                 %Use: \Racionals por Notation
    \DeclareMathOperator \Reals        {\mathbb{R}}                 %Use: \Reals por Notation
    \DeclareMathOperator \Complexs     {\mathbb{C}}                 %Use: \Complex por Notation
    \DeclareMathOperator \GenericField {\mathbb{F}}                 %Use: \GenericField por Notation
    \DeclareMathOperator \VectorSet    {\mathbb{V}}                 %Use: \VectorSet por Notation
    \DeclareMathOperator \SubVectorSet {\mathbb{W}}                 %Use: \SubVectorSet por Notation
    \DeclareMathOperator \Polynomials  {\mathbb{P}}                 %Use: \Polynomials por Notation
    \DeclareMathOperator \VectorSpace  {\VectorSet_{\GenericField}} %Use: \VectorSpace por Notation
    \DeclareMathOperator \LinealTransformation {\mathcal{T}}        %Use: \LinealTransformation for a cool T
    \DeclareMathOperator \LinTrans {\mathcal{T}}                    %Use: \LinTrans for a cool T


    % =====  CONTAINERS   ===============
    \newcommand{\Set}[1]    {\left\{ \; #1 \; \right\}}             %Use: \Set {Info} for INTELLIGENT space 
    \newcommand{\bigSet}[1] {\big\{  \; #1 \; \big\}}               %Use: \bigSet  {Info} for space 
    \newcommand{\BigSet}[1] {\Big\{  \; #1 \; \Big\}}               %Use: \BigSet  {Info} for space 
    \newcommand{\biggSet}[1]{\bigg\{ \; #1 \; \bigg\}}              %Use: \biggSet {Info} for space 
    \newcommand{\BiggSet}[1]{\Bigg\{ \; #1 \; \Bigg\}}              %Use: \BiggSet {Info} for space 
    
    \newcommand{\Brackets}[1]    {\left[ #1 \right]}                %Use: \Brackets {Info} for INTELLIGENT space
    \newcommand{\bigBrackets}[1] {\big[ \; #1 \; \big]}             %Use: \bigBrackets  {Info} for space 
    \newcommand{\BigBrackets}[1] {\Big[ \; #1 \; \Big]}             %Use: \BigBrackets  {Info} for space 
    \newcommand{\biggBrackets}[1]{\bigg[ \; #1 \; \bigg]}           %Use: \biggBrackets {Info} for space 
    \newcommand{\BiggBrackets}[1]{\Bigg[ \; #1 \; \Bigg]}           %Use: \BiggBrackets {Info} for space 
    
    \newcommand{\Wrap}[1]    {\left( #1 \right)}                    %Use: \Wrap {Info} for INTELLIGENT space
    \newcommand{\bigWrap}[1] {\big( \; #1 \; \big)}                 %Use: \bigBrackets  {Info} for space 
    \newcommand{\BigWrap}[1] {\Big( \; #1 \; \Big)}                 %Use: \BigBrackets  {Info} for space 
    \newcommand{\biggWrap}[1]{\bigg( \; #1 \; \bigg)}               %Use: \biggBrackets {Info} for space 
    \newcommand{\BiggWrap}[1]{\Bigg( \; #1 \; \Bigg)}               %Use: \BiggBrackets {Info} for space 
    
    \newcommand{\Generate}[1]{\left\langle #1 \right\rangle}        %Use: \Generate {Info} <>
    \newcommand{\Floor}[1]{\left \lfloor #1 \right \rfloor}         %Use: \Floor {Info} for floor 
    \newcommand{\Ceil}[1]{\left \lceil #1 \right \rceil }           %Use: \Ceil {Info} for ceil

    % =====  BETTERS MATH COMMANDS   =====
    \newcommand{\pfrac}[2]{\Wrap{\dfrac{#1}{#2}}}                   %Use: Put fractions in parentesis

    % =========================================
    % ====   LINEAL ALGEBRA & VECTORS    ======
    % =========================================

    % ===== UNIT VECTORS  ================
    \newcommand{\hati} {\hat{\imath}}                               %Use: \hati for unit vector    
    \newcommand{\hatj} {\hat{\jmath}}                               %Use: \hatj for unit vector    
    \newcommand{\hatk} {\hat{k}}                                    %Use: \hatk for unit vector

    % ===== FN LINEAL TRANSFORMATION  ====
    \newcommand{\FnLinTrans}[1]{\mathcal{T}\Wrap{#1}}               %Use: \FnLinTrans for a cool T
    \newcommand{\VecLinTrans}[1]{\mathcal{T}\pVector{#1}}           %Use: \LinTrans for a cool T
    \newcommand{\FnLinealTransformation}[1]{\mathcal{T}\Wrap{#1}}   %Use: \FnLinealTransformation

    % ===== MAGNITUDE  ===================
    \newcommand{\abs}[1]{\left\lvert #1 \right\lvert}               %Use: \abs{expression} for |x|
    \newcommand{\Abs}[1]{\left\lVert #1 \right\lVert}               %Use: \Abs{expression} for ||x||
    \newcommand{\Mag}[1]{\left| #1 \right|}                         %Use: \Mag {Info} 
    
    \newcommand{\bVec}[1]{\mathbf{#1}}                              %Use for bold type of vector
    \newcommand{\lVec}[1]{\overrightarrow{#1}}                      %Use for a long arrow over a vector
    \newcommand{\uVec}[1]{\mathbf{\hat{#1}}}                        %Use: Unitary Vector Example: $\uVec{i}

    % ===== ALL FOR DOT PRODUCT  =========
    \makeatletter                                                   %WTF! IS THIS
    \newcommand*\dotP{\mathpalette\dotP@{.5}}                       %Use: \dotP for dot product
    \newcommand*\dotP@[2] {\mathbin {                               %WTF! IS THIS            
        \vcenter{\hbox{\scalebox{#2}{$\m@th#1\bullet$}}}}           %WTF! IS THIS
    }                                                               %WTF! IS THIS
    \makeatother                                                    %WTF! IS THIS

    % === WRAPPERS FOR COLUMN VECTOR ===
    \newcommand{\pVector}[1]                                        %Use: \pVector {Matrix Notation} use parentesis
        { \ensuremath{\begin{pmatrix}#1\end{pmatrix}} }             %Example: \pVector{a\\b\\c} or \pVector{a&b&c} 
    \newcommand{\lVector}[1]                                        %Use: \lVector {Matrix Notation} use a abs 
        { \ensuremath{\begin{vmatrix}#1\end{vmatrix}} }             %Example: \lVector{a\\b\\c} or \lVector{a&b&c} 
    \newcommand{\bVector}[1]                                        %Use: \bVector {Matrix Notation} use a brackets 
        { \ensuremath{\begin{bmatrix}#1\end{bmatrix}} }             %Example: \bVector{a\\b\\c} or \bVector{a&b&c} 
    \newcommand{\Vector}[1]                                         %Use: \Vector {Matrix Notation} no parentesis
        { \ensuremath{\begin{matrix}#1\end{matrix}} }               %Example: \Vector{a\\b\\c} or \Vector{a&b&c}

    % === MAKE MATRIX BETTER  =========
    \makeatletter                                                   %Example: \begin{matrix}[cc|c]
    \renewcommand*\env@matrix[1][*\c@MaxMatrixCols c] {             %WTF! IS THIS
        \hskip -\arraycolsep                                        %WTF! IS THIS
        \let\@ifnextchar\new@ifnextchar                             %WTF! IS THIS
        \array{#1}                                                  %WTF! IS THIS
    }                                                               %WTF! IS THIS
    \makeatother                                                    %WTF! IS THIS

    % =========================================
    % =======   FAMOUS FUNCTIONS   ============
    % =========================================

    % == TRIGONOMETRIC FUNCTIONS  ====
    \newcommand{\Cos}[1] {\cos\Wrap{#1}}                            %Simple wrappers
    \newcommand{\Sin}[1] {\sin\Wrap{#1}}                            %Simple wrappers
    \newcommand{\Tan}[1] {tan\Wrap{#1}}                             %Simple wrappers
    
    \newcommand{\Sec}[1] {sec\Wrap{#1}}                             %Simple wrappers
    \newcommand{\Csc}[1] {csc\Wrap{#1}}                             %Simple wrappers
    \newcommand{\Cot}[1] {cot\Wrap{#1}}                             %Simple wrappers

    % === COMPLEX ANALYSIS TRIG ======
    \newcommand \Cis[1]  {\Cos{#1} + i \Sin{#1}}                    %Use: \Cis for cos(x) + i sin(x)
    \newcommand \pCis[1] {\Wrap{\Cis{#1}}}                          %Use: \pCis for the same with parantesis
    \newcommand \bCis[1] {\Brackets{\Cis{#1}}}                      %Use: \bCis for the same with Brackets


    % =========================================
    % ===========     CALCULUS     ============
    % =========================================

    % ====== TRANSFORMS =============
    \newcommand{\FourierT}[1]{\mathscr{F} \left\{ #1 \right\} }     %Use: \FourierT {Funtion}
    \newcommand{\InvFourierT}[1]{\mathscr{F}^{-1}\left\{#1\right\}} %Use: \InvFourierT {Funtion}

    % ====== DERIVATIVES ============
    \newcommand \MiniDerivate[1][x] {\dfrac{d}{d #1}}               %Use: \MiniDerivate[var] for simple use [var]
    \newcommand \Derivate[2] {\dfrac{d \; #1}{d #2}}                %Use: \Derivate [f(x)][x]
    \newcommand \MiniUpperDerivate[2] {\dfrac{d^{#2}}{d#1^{#2}}}    %Mini Derivate High Orden Derivate -- [x][pow]
    \newcommand \UpperDerivate[3] {\dfrac{d^{#3} \; #1}{d#2^{#3}}}  %Complete High Orden Derivate -- [f(x)][x][pow]
    
    \newcommand \MiniPartial[1][x] {\dfrac{\partial}{\partial #1}}  %Use: \MiniDerivate for simple use [var]
    \newcommand \Partial[2] {\dfrac{\partial \; #1}{\partial #2}}   %Complete Partial Derivate -- [f(x)][x]
    \newcommand \MiniUpperPartial[2]                                %Mini Derivate High Orden Derivate -- [x][pow] 
        {\dfrac{\partial^{#2}}{\partial #1^{#2}}}                   %Mini Derivate High Orden Derivate
    \newcommand \UpperPartial[3]                                    %Complete High Orden Derivate -- [f(x)][x][pow]
        {\dfrac{\partial^{#3} \; #1}{\partial#2^{#3}}}              %Use: \UpperDerivate for simple use

    \DeclareMathOperator \Evaluate  {\Big|}                         %Use: \Evaluate por Notation

    % =========================================
    % ========    GENERAL STYLE     ===========
    % =========================================
    
    % =====  COLORS ==================
    \definecolor{RedMD}{HTML}{F44336}                               %Use: Color :D        
    \definecolor{Red100MD}{HTML}{FFCDD2}                            %Use: Color :D        
    \definecolor{Red200MD}{HTML}{EF9A9A}                            %Use: Color :D        
    \definecolor{Red300MD}{HTML}{E57373}                            %Use: Color :D        
    \definecolor{Red700MD}{HTML}{D32F2F}                            %Use: Color :D 

    \definecolor{PurpleMD}{HTML}{9C27B0}                            %Use: Color :D        
    \definecolor{Purple100MD}{HTML}{E1BEE7}                         %Use: Color :D        
    \definecolor{Purple200MD}{HTML}{EF9A9A}                         %Use: Color :D        
    \definecolor{Purple300MD}{HTML}{BA68C8}                         %Use: Color :D        
    \definecolor{Purple700MD}{HTML}{7B1FA2}                         %Use: Color :D 

    \definecolor{IndigoMD}{HTML}{3F51B5}                            %Use: Color :D        
    \definecolor{Indigo100MD}{HTML}{C5CAE9}                         %Use: Color :D        
    \definecolor{Indigo200MD}{HTML}{9FA8DA}                         %Use: Color :D        
    \definecolor{Indigo300MD}{HTML}{7986CB}                         %Use: Color :D        
    \definecolor{Indigo700MD}{HTML}{303F9F}                         %Use: Color :D 

    \definecolor{BlueMD}{HTML}{2196F3}                              %Use: Color :D        
    \definecolor{Blue100MD}{HTML}{BBDEFB}                           %Use: Color :D        
    \definecolor{Blue200MD}{HTML}{90CAF9}                           %Use: Color :D        
    \definecolor{Blue300MD}{HTML}{64B5F6}                           %Use: Color :D        
    \definecolor{Blue700MD}{HTML}{1976D2}                           %Use: Color :D        
    \definecolor{Blue900MD}{HTML}{0D47A1}                           %Use: Color :D  

    \definecolor{CyanMD}{HTML}{00BCD4}                              %Use: Color :D        
    \definecolor{Cyan100MD}{HTML}{B2EBF2}                           %Use: Color :D        
    \definecolor{Cyan200MD}{HTML}{80DEEA}                           %Use: Color :D        
    \definecolor{Cyan300MD}{HTML}{4DD0E1}                           %Use: Color :D        
    \definecolor{Cyan700MD}{HTML}{0097A7}                           %Use: Color :D        
    \definecolor{Cyan900MD}{HTML}{006064}                           %Use: Color :D 

    \definecolor{TealMD}{HTML}{009688}                              %Use: Color :D        
    \definecolor{Teal100MD}{HTML}{B2DFDB}                           %Use: Color :D        
    \definecolor{Teal200MD}{HTML}{80CBC4}                           %Use: Color :D        
    \definecolor{Teal300MD}{HTML}{4DB6AC}                           %Use: Color :D        
    \definecolor{Teal700MD}{HTML}{00796B}                           %Use: Color :D        
    \definecolor{Teal900MD}{HTML}{004D40}                           %Use: Color :D 

    \definecolor{GreenMD}{HTML}{4CAF50}                             %Use: Color :D        
    \definecolor{Green100MD}{HTML}{C8E6C9}                          %Use: Color :D        
    \definecolor{Green200MD}{HTML}{A5D6A7}                          %Use: Color :D        
    \definecolor{Green300MD}{HTML}{81C784}                          %Use: Color :D        
    \definecolor{Green700MD}{HTML}{388E3C}                          %Use: Color :D        
    \definecolor{Green900MD}{HTML}{1B5E20}                          %Use: Color :D

    \definecolor{AmberMD}{HTML}{FFC107}                             %Use: Color :D        
    \definecolor{Amber100MD}{HTML}{FFECB3}                          %Use: Color :D        
    \definecolor{Amber200MD}{HTML}{FFE082}                          %Use: Color :D        
    \definecolor{Amber300MD}{HTML}{FFD54F}                          %Use: Color :D        
    \definecolor{Amber700MD}{HTML}{FFA000}                          %Use: Color :D        
    \definecolor{Amber900MD}{HTML}{FF6F00}                          %Use: Color :D

    \definecolor{BlueGreyMD}{HTML}{607D8B}                          %Use: Color :D        
    \definecolor{BlueGrey100MD}{HTML}{CFD8DC}                       %Use: Color :D        
    \definecolor{BlueGrey200MD}{HTML}{B0BEC5}                       %Use: Color :D        
    \definecolor{BlueGrey300MD}{HTML}{90A4AE}                       %Use: Color :D        
    \definecolor{BlueGrey700MD}{HTML}{455A64}                       %Use: Color :D        
    \definecolor{BlueGrey900MD}{HTML}{263238}                       %Use: Color :D        

    \definecolor{DeepPurpleMD}{HTML}{673AB7}                        %Use: Color :D

    \newcommand{\Color}[2]{\textcolor{#1}{#2}}                      %Simple color environment
    \newenvironment{ColorText}[1]                                   %Use: \begin{ColorText}
        { \leavevmode\color{#1}\ignorespaces }                      %That's is!

    % =====  CODE EDITOR =============
    \lstdefinestyle{CompilandoStyle} {                              %This is Code Style
        backgroundcolor     = \color{BlueGrey900MD},                %Background Color  
        basicstyle          = \tiny\color{white},                   %Style of text
        commentstyle        = \color{BlueGrey200MD},                %Comment style
        stringstyle         = \color{Green300MD},                   %String style
        keywordstyle        = \color{Blue300MD},                    %keywords style
        numberstyle         = \tiny\color{TealMD},                  %Size of a number
        frame               = shadowbox,                            %Adds a frame around the code
        breakatwhitespace   = true,                                 %Style   
        breaklines          = true,                                 %Style   
        showstringspaces    = false,                                %Hate those spaces                  
        breaklines          = true,                                 %Style                   
        keepspaces          = true,                                 %Style                   
        numbers             = left,                                 %Style                   
        numbersep           = 10pt,                                 %Style 
        xleftmargin         = \parindent,                           %Style 
        tabsize             = 4,                                    %Style
        inputencoding       = utf8/latin1                           %Allow me to use special chars
    }
 
    \lstset{style = CompilandoStyle}                                %Use this style


% =====================================================
% ============        COVER PAGE       ================
% =====================================================
\begin{document}
\begin{titlepage}
    
    % ============ TITLE PAGE STYLE  ================
    \definecolor{TitlePageColor}{cmyk}{1,.60,0,.40}                 %Simple colors
    \definecolor{ColorSubtext}{cmyk}{1,.50,0,.10}                   %Simple colors
    \newgeometry{left=0.25\textwidth}                               %Defines an Offset
    \pagecolor{TitlePageColor}                                      %Make it this Color to page
    \color{white}                                                   %General things should be white

    % ===== MAKE SOME SPACE =========
    \vspace                                                         %Give some space
    \baselineskip                                                   %But we need this to up command

    % ============ NAME OF THE PROJECT  ============
    \makebox[0pt][l]{\rule{1.3\textwidth}{3pt}}                     %Make a cool line
    
    \href{https://compilandoconocimiento.com/}                      %Link to project
    {\textbf{\textsc{\Huge ESCOM - IPN}}}\\[2.7cm]                  %Name of project   

    % ============ NAME OF THE BOOK  ===============
    \href{https://github.com/CompilandoConocimiento/}               %Link to Author
    {\fontsize{65}{78}\selectfont 
        \textbf{Análisis con Big O Notation}}\\[0.5cm]              %Name of the book
    \textcolor{ColorSubtext}
        {\textsc{\Huge Análisis de Algoritmos 3CM3}}\\[2cm]         %Name of the general theme
    
    \vfill                                                          %Fill the space

    \begin{figure}[h]
        \includegraphics[width=0.25\textwidth]{Photo}
    \end{figure}
    
    % ============ NAME OF THE AUTHOR  =============
    \href{ProjectAuthorLink}                                        %Link to Author
    {\LARGE \textsf{Oscar Andrés Rosas Hernandez}}                  %Author

    % ===== MAKE SOME SPACE =========
    \vspace                                                         %Give some space
    \baselineskip                                                   %But we need this to up command
    
    {\large \textsf{Abril 2018}}                                    %Date

\end{titlepage}


% =====================================================
% ==========      RESTORE TO DOCUMENT      ============
% =====================================================
\restoregeometry                                                    %Restores the geometry
\nopagecolor                                                        %Use to restore the color to white


% =====================================================
% ========                INDICE              =========
% =====================================================
\tableofcontents{}
\label{sec:Index}

\clearpage



% ============================================================
% =================          ALGORITMOS      =================
% ============================================================
\clearpage
\section{Algoritmo 1}


    % ============================================================
    % =================          ALGORITMOS 1    =================
    % ============================================================
    \subsection{Análisis Visual}

        \begin{figure}[h]
            \centering
            \includegraphics[width=0.95\textwidth]{Algoritmo1}
            \caption{Análisis Visual del Algoritmo}
        \end{figure}

    % ============================================================
    % ============          EXPLICACION          =================
    % ============================================================
    \vspace{1em}
    \subsection{Explicación}


        En esta caso el algoritmo es bastante sencillo de analizar, por un lado tenemos
        2 bubles for anidaddos, cada uno de ellos desde cero hasta $n - 1$, por lo tanto se repetiran cada
        uno $n$ veces.

        Ahora lo que esta dentro de los ciclos son 3 operaciones primitivas, por lo tanto todas se aproximan
        a una cota $O(1)$.

        Finalmente podemos decir que todo el algoritmo en general tiene un orden de $O(n^2)$




% ============================================================
% =================          ALGORITMO 2     =================
% ============================================================
\clearpage
\section{Algoritmo 2}


    % ============================================================
    % =================          ALGORITMOS      =================
    % ============================================================
    \subsection{Análisis Visual}

        \begin{figure}[h]
            \centering
            \includegraphics[width=0.95\textwidth]{Algoritmo2}
            \caption{Análisis Visual del Algoritmo}
        \end{figure}

    % ============================================================
    % ============          EXPLICACION          =================
    % ============================================================
    \vspace{1em}
    \subsection{Explicación}

        La primera instrucción es primitiva, por lo tanto es constante,
        despúes tenemos un polinomio que se ejecuta $n$ veces,
        dentro tenemos una operación constante, por lo tanto es bastante
        sencillo ver que el algoritmo es en general $O(n)$



% ============================================================
% =================          ALGORITMO 3     =================
% ============================================================
\clearpage
\section{Algoritmo 3}


    % ============================================================
    % =================          ALGORITMOS      =================
    % ============================================================
    \subsection{Análisis Visual}

        \begin{figure}[h]
            \centering
            \includegraphics[width=0.95\textwidth]{Algoritmo3}
            \caption{Análisis Visual del Algoritmo}
        \end{figure}

    % ============================================================
    % ============          EXPLICACION          =================
    % ============================================================
    \vspace{1em}
    \subsection{Explicación}

        Tenemos 3 bucles for anidados, cada uno de ellos desde $1$ hasta $n$, 
        por lo tanto tenemos $n$ operaciones dentro de cada for.

        Y dentro de los 3 fors anidados tenemos algo muy loco, una asignación y multiplicación
        por lo tanto tenemos un algoritmo en total de $0(n^3)$ cada exponente por el bucle for







% ============================================================
% =================          ALGORITMO 4     =================
% ============================================================
\clearpage
\section{Algoritmo 4}


    % ============================================================
    % =================          ALGORITMOS      =================
    % ============================================================
    \subsection{Análisis Visual}

        \begin{figure}[h]
            \centering
            \includegraphics[width=0.95\textwidth]{Algoritmo4}
            \caption{Análisis Visual del Algoritmo}
        \end{figure}

    % ============================================================
    % ============          EXPLICACION          =================
    % ============================================================
    \vspace{1em}
    \subsection{Explicación}

        Ok, este esta mucho mas bueno que los otros, primero, dentro del
        ciclo while tenemos operaciones primitivas, por lo tanto estas serán
        constantes, ahora...

        ¿Cuántas veces se ejecuta el ciclo While? Este se ejecuta $n - 2$ veces 
        porque en cada ciclo del while, $n$ va decreciendo de uno a uno, hasta llegar
        a $2$.

        Ahora, en la parte superior tenemos $2$ instrucciones primitivas, que nos toman
        siempre la misma cantidad de tiempo asintótico.

        Por lo tanto en general podemos decir que este algoritmo en general es linea
        con respecto a $n$, es decir $0(n)$ pues $O(n) > O(1)$.






    % ============================================================
    % =================          ALGORITMOS      =================
    % ============================================================
    \clearpage
    \subsection{Algoritmo 2}
    \begin{SmallIndentation}[0em]
            
        \begin{lstlisting}[language=C, gobble=12, basicstyle=\small\color{white}]
            polinomio = 0;                                  
                for(i = 0; i <= n; i++) {
                    polinomio = polinomio * z + A[n - i];       
                }
        \end{lstlisting}   



        \begin{enumerate}
            \item
                Veamos primero la linea: polinomio = polinomio * z + A[n - i];

                Esta esta algo enconglomerada, pues primero es:
                \begin{itemize}
                    \item 1 por $n - i$
                    \item 1 por $polinomio * z$
                    \item 1 por suma de ambas
                    \item 1 por asignación
                \end{itemize}

                Por lo tanto cuesta 4 unidades


            \item Todo lo anterior esta encerrado en un for generico, 
                asi que vamos a calcularlo:
                \begin{align*}
                    Cost 
                        &= 3 + Cost(Start) + Cost(End) + NumInnerFor(Cost(End) + 4 + Cost(DoSomething))     \\
                        &= 3 + 0 + 0 + NumInnerFor(0 + 4 + Cost(DoSomething))                               \\
                        &= 3 + 0 + 0 + NumInnerFor(0 + 4 + 4)                                               \\
                        &= 3 + NumInnerFor(8)                                                             
                \end{align*}

                Ahora sabemos que 
                $NumInnerFor 
                    = \Floor{\frac{End - Start}{Jump} + 1}
                    = \Floor{\frac{n - 0}{1} + 1}
                    = \Floor{\frac{n}{1} + 1}
                    = n+1$

                Por lo tanto en general $Cost = 8(n+1)+3 = 8n - 12$

        \end{enumerate}

        Finalmente tenemos que:
        \begin{itemize}
            \item $Cost_{Temporal} = 8n + 12$
            \item $Cost_{Espacial} = n + 3$

                Porque pues usamos un arreglo de $n + 1$ elementos (para hacer A[n]), 
                espacio para temp y espacio para 1 interador
        \end{itemize}
            
    \end{SmallIndentation}



    % ============================================================
    % =================          ALGORITMOS      =================
    % ============================================================
    \clearpage
    \subsection{Algoritmo 4}
    \begin{SmallIndentation}[0em]
            
        \begin{lstlisting}[language=C, gobble=12, basicstyle=\small\color{white}]
            anterior = 1;                   
            actual = 1;                     
            while (n > 2) {
                aux = anterior + actual;    
                anterior = actual;          
                actual = aux;               
                n = n - 1;
            }
        \end{lstlisting}   

        Antes que nada, vamos a ponerlo en un modo de for, para que todo sea mas sencillo:
        \begin{lstlisting}[language=C, gobble=12, basicstyle=\small\color{white}]
            anterior = 1;                   
            actual = 1;  

            for (i = 3; i <= n; i++) {                  
                aux = anterior + actual;    
                anterior = actual;          
                actual = aux;               
            }
        \end{lstlisting} 

        \begin{enumerate}
            \item
                Veamos primero lo que esta dentro del for

                Esta esta algo enconglomerada, pues primero es:
                \begin{itemize}
                    \item 1 por suma y otro por asignación
                    \item 1 por asignación
                    \item 1 por asignación
                \end{itemize}

                Por lo tanto cuesta 4 unidades


            \item Todo lo anterior esta encerrado en un for generico, 
                asi que vamos a calcularlo:
                \begin{align*}
                    Cost 
                        &= 3 + Cost(Start) + Cost(End) + NumInnerFor(Cost(End) + 4 + Cost(DoSomething))     \\
                        &= 3 + 0 + 0 + NumInnerFor(0+ 4 + 4)                \\
                        &= 3 + NumInnerFor(8)                
                \end{align*}

                Ahora sabemos que 
                $NumInnerFor 
                    = \Floor{\frac{End - Start}{Jump} + 1}
                    = \Floor{\frac{n - 3}{1} + 1}
                    = \Floor{\frac{n-3}{1} + 1}
                    = n-2$

                Por lo tanto en general $Cost = 8(n-2)+3 = 8n - 13$

        \end{enumerate}

        Finalmente tenemos que:
        \begin{itemize}
            \item $Cost_{Temporal} = 8n - 13 + 4 = 8n - 9$
            \item $Cost_{Espacial} = 3$

                Porque pues usamos 3 variables nada mas, 2 normales y 1 interador
        \end{itemize}
            
    \end{SmallIndentation}




    % ============================================================
    % =================          ALGORITMOS      =================
    % ============================================================
    \clearpage
    \subsection{Algoritmo 5}
    \begin{SmallIndentation}[0em]
            
        \begin{lstlisting}[language=C, gobble=12, basicstyle=\small\color{white}]
            for (i = n - 1, j = 0; i >= 0; i--, j++)
                s2[j] = s[i];

            for (i = 0, i < n; i++)
                s[i] = s2[i];
        \end{lstlisting}

        Ok, esta medio raro el codigo, vamos a ponerlo bonito:
        \begin{lstlisting}[language=C, gobble=12, basicstyle=\small\color{white}]
            j = 0;
            for (i = 0; i <= n - 1; i++) {
                s2[j] = s[n - 1 - i];
                j++;
            }

            for (i = 0, i <= n -1; i++)
                s[i] = s2[i];
        \end{lstlisting}

        Ok, esta esta muy sencilla, porque son sencillamente 2 fors

        \begin{enumerate}

            \item
                Internamente lo que pasa dentro del primer for es que la primera
                linea son 2 operaciones aritmetica y 1 asignación en la primera
                linea y una operación y otra asignación, por lo tanto calculamos
                5 operaciones

            \item Todo lo anterior esta encapsulado en un for generico, 
                asi que vamos a calcularlo:
                \begin{align*}
                    Cost 
                        &= 3 + Cost(Start) + Cost(End) + NumInnerFor(Cost(End) + 4 + Cost(DoSomething))     \\
                        &= 3 + 0 + 0 + NumInnerFor(0 + 4 + 3)     \\
                        &= 3 + NumInnerFor(3)     \\
                \end{align*}

                Ahora sabemos que 
                $NumInnerFor 
                    = \Floor{\frac{End - Start}{Jump} + 1}
                    = \Floor{\frac{n - 1- 0}{1} + 1}
                    = \Floor{\frac{n-1}{1} + 1}
                    = n$

                Por lo tanto en general $Cost = n(7)+3 = 7n + 3$

            \item
                La siguiente parte del algoritmo es otro pequeño y simple for generico:
                \begin{align*}
                    Cost 
                        &= 3 + Cost(Start) + Cost(End) + NumInnerFor(Cost(End) + 4 + Cost(DoSomething))     \\
                        &= 3 + 0 + 0 + NumInnerFor(0 + 4 + 1)     \\
                        &= 3 + NumInnerFor(5)     \\
                \end{align*}

                Ahora sabemos que 
                $NumInnerFor 
                    = \Floor{\frac{End - Start}{Jump} + 1}
                    = \Floor{\frac{n - 1- 0}{1} + 1}
                    = \Floor{\frac{n-1}{1} + 1}
                    = n$

        \end{enumerate}

        Finalmente tenemos que:
        \begin{itemize}
            \item $Cost_{Temporal} = 5n + 3 + 7n + 3 = 12n + 6$
            \item $Cost_{Espacial} = 2n + 2$

                Porque pues usamos 2 iterados, y 2 arreglos mínimo n elementos
        \end{itemize}
            
    \end{SmallIndentation}



% ============================================================
% =================          ALGORITMO 5     =================
% ============================================================
\clearpage
\section{Algoritmo 5}


    % ============================================================
    % =================          ALGORITMOS      =================
    % ============================================================
    \subsection{Análisis Visual}

        \begin{figure}[h]
            \centering
            \includegraphics[width=0.95\textwidth]{Algoritmo5}
            \caption{Análisis Visual del Algoritmo}
        \end{figure}

    % ============================================================
    % ============          EXPLICACION          =================
    % ============================================================
    \vspace{1em}
    \subsection{Explicación}

        Antes que nada, no te preocupes, no es que haya modificado mucho el algoritmo, lo unico
        que hice fue hacer que los bucles crezcan, y separar un poco a la variable $j$.

        Ahora si, tenemos 2 ciclos for, por lo tanto la cota $O()$ estará dada por la suma de 
        ambos costos de los ciclos:
        \begin{itemize}
            \item 
                En el primer ciclo tenemos solo operaciones primitivas dentro del for,
                y este ira de 0 hasta $n -1$ por lo tanto podemos concluir que el primer ciclo
                tiene un costo de $O(n)$
            \item 
                En el segundo ciclo tenemos solo operaciones primitivas dentro del for,
                y este ira también de 0 hasta $n -1$ por lo tanto podemos concluir que el segundo ciclo
                tiene un costo de $O(n)$
        \end{itemize}

        Por lo tanto tenemos que el costo total es $O(n) + O(n) = O(n)$



% ============================================================
% ===============    ALGORITMO  6 - 9        =================
% ============================================================
\clearpage
\section{Algoritmo 6 - 9}


    % ============================================================
    % =================          ALGORITMOS      =================
    % ============================================================
    \subsection{Análisis Visual}

        \begin{figure}[h]
            \centering
            \includegraphics[width=0.85\textwidth]{Algoritmo6}
            \caption{Análisis Visual del Algoritmo}
        \end{figure}

    % ============================================================
    % ============          EXPLICACION          =================
    % ============================================================
    \vspace{1em}
    \subsection{Explicación}

        Ahora si estamos empezando a ver algoritmos mucho más interesantes,  

        Ahora si, tenemos 2 bloques de codigo dentro de la función, por lo tanto la cota $O()$
        de la función completa estará dada por la suma de ambos costos:
        \begin{itemize}
            \item 
                Bloque de código 1, en el peor de los casos (despúes de todo de eso también se
                trata $O()$) tenemos operaciones primitivas, de hecho, sin importar como sea
                núestro vector en si, por lo tanto el primer bloque tiene un costo de $O(1)$
            \item 
                En el segundo bloque de código esta un poco más interesante, tenemos una variable
                $i$ que ira creciendo desde 3 hasta $n$, por lo tanto se ejecutará $n -3$
                veces el código interno, pero lo curioso es que dentro del bucle while
                solo tenemos operaciones primitivas, sin importar como sea nuestra información
                el costo del interior del bucle será $O(1)$.

                Por lo tanto el costo de este bloque será de $O(n)$
        \end{itemize}

        Por lo tanto tenemos que el costo total es $O(n) + O(1) = O(n)$



% ============================================================
% ===============    ALGORITMO  7 - 10       =================
% ============================================================
\clearpage
\section{Algoritmo 7 - 10}


    % ============================================================
    % =================          ALGORITMOS      =================
    % ============================================================
    \subsection{Análisis Visual}

        \begin{figure}[h]
            \centering
            \includegraphics[width=0.95\textwidth]{Algoritmo7}
            \caption{Análisis Visual del Algoritmo}
        \end{figure}

    % ============================================================
    % ============          EXPLICACION          =================
    % ============================================================
    \vspace{1em}
    \subsection{Explicación}

        Nuestro primero algoritmo de ordenamiento, ¡Qué emoción!

        Vamos de adentro hacia afuera, por un lado, primero, en el peor de los casos
        nuestro algoritmo siempre entrará dentro del if, dentro solo tenemos operaciones
        básicas, por lo tanto incluso si estamos lo que esta dentro de segundo for tiene un costo
        constante, es decir $O(1)$.

        Ahora este código constante se hará primero desde $0$ hasta $n$, luego de $1$ hasta $n$, y 
        así, por lo tanto podemos ver que si lo ordenamos al reves tenemos que se ejecutará $\sum_{i=1}^n i$
        veces pero lo lindo es que ese resultado es muy conocido, entonces entre los dos fors el código
        constante se hará $\frac{n(n+1)}{2}$, que si aplicamos el análisis sintótico tenemos que tiene
        un costo de $O(n^2)$ al final y al cabo.



% ============================================================
% ===============    ALGORITMO  8 - 11       =================
% ============================================================
\clearpage
\section{Algoritmo 8 - 11}


    % ============================================================
    % =================          ALGORITMOS      =================
    % ============================================================
    \subsection{Análisis Visual}

        \begin{figure}[h]
            \centering
            \includegraphics[width=0.95\textwidth]{Algoritmo8}
            \caption{Análisis Visual del Algoritmo}
        \end{figure}

    % ============================================================
    % ============          EXPLICACION          =================
    % ============================================================
    \vspace{1em}
    \subsection{Explicación}

        Para obtener el $O()$ tenemos que ver como es que se comporta nuestro algoritmo
        en el peor de los casos mientras nuestro tamaño del problema crece, así que vamos
        primero a recordar lo que dije en el trabajo pasado:

        Ok, tengo que admitir que este es un problema conocido, es el clásico algoritmo de Euclides
        y hay un caso muy feo para ese algoritmo, y es que sean dos número de Fibonacci consecutivos.

        Vamos a dar una explicación de porque:
        Por definición $F_{k} = F_{k -1} + F_{k - 2}$, por lo tanto cuanto lo divides tienes que 
        que el cociente siempre sera 1 en cada iteración, por lo tanto tardaremos $k$ divisiones en 
        llegar a $k$, y como los números de Fibonacci se aproximan a $\phi^k$ entonces
        tenemos que el peor caso será $f(n) = \log{\phi}(n)$

        Así que ahora, conociendo la función complejidad del peor caso es trivial ver que tiene
        un costo de $O(\log_{\phi}(n))$


% ============================================================
% ===============    ALGORITMO  9 - 12       =================
% ============================================================
\clearpage
\section{Algoritmo 9 - 12}


    % ============================================================
    % =================          ALGORITMOS      =================
    % ============================================================
    \subsection{Análisis Visual}

        \begin{figure}[h]
            \centering
            \includegraphics[width=0.95\textwidth]{Algoritmo9}
            \caption{Análisis Visual del Algoritmo}
        \end{figure}

    % ============================================================
    % ============          EXPLICACION          =================
    % ============================================================
    \vspace{1em}
    \subsection{Explicación}

        Ok, voy creo que ya vamos teniendo experiencia para ver que las 
        primeras 2 intrucciones son primitivas por lo tanto no importa mucho
        que es lo que pase con ellas, sino lo que pasa dentro del ciclo.

        Ahora, claro, en el mejor de los casos la variable de cambio podría hacer
        que salieramos repentinamente del ciclo, pero en el peor de los casos, que es
        lo que estamos haciendo esta variable nunca pinta de nada.

        Y si eliminamos esa variable / bandera tenemos que volvemos al mismo algoritmo
        de burbuja, con sus sumas de Gauss que ya lo resolví en otros ejercicios.

        Así que creo que con estos argumentos tiene mucho sentido que este algoritmo
        siga siendo $O(n^2)$



% ============================================================
% ===============    ALGORITMO  10 - 13      =================
% ============================================================
\clearpage
\section{Algoritmo 10 - 13}


    % ============================================================
    % =================          ALGORITMOS      =================
    % ============================================================
    \subsection{Análisis Visual}

        \begin{figure}[h]
            \centering
            \includegraphics[width=0.95\textwidth]{Algoritmo10}
            \caption{Análisis Visual del Algoritmo}
        \end{figure}

    % ============================================================
    % ============          EXPLICACION          =================
    % ============================================================
    \vspace{1em}
    \subsection{Explicación}

        Nuestro segundo algoritmo de ordenamiento, ¡Qué emoción!

        Vamos de adentro hacia afuera, por un lado, primero, en el peor de los casos
        nuestro algoritmo siempre entrará dentro del if, dentro solo tenemos operaciones
        básicas, por lo tanto incluso si estamos lo que esta dentro de segundo for tiene un costo
        constante, es decir $O(1)$.

        Ahora este código constante se hará primero desde $0$ hasta $n$, luego de $1$ hasta $n$, y 
        así, por lo tanto podemos ver que si lo ordenamos al reves tenemos que se ejecutará $\sum_{i=1}^n i$
        veces pero lo lindo es que ese resultado es muy conocido, entonces entre los dos fors el código
        constante se hará $\frac{n(n+1)}{2}$, que si aplicamos el análisis sintótico tenemos que tiene
        un costo de $O(n^2)$ al final y al cabo.




% ============================================================
% ===============    ALGORITMO  11 - 14      =================
% ============================================================
\clearpage
\section{Algoritmo 11 - 14}


    % ============================================================
    % =================          ALGORITMOS      =================
    % ============================================================
    \subsection{Análisis Visual}

        \begin{figure}[h]
            \centering
            \includegraphics[width=0.95\textwidth]{Algoritmo11}
            \caption{Análisis Visual del Algoritmo}
        \end{figure}

    % ============================================================
    % ============          EXPLICACION          =================
    % ============================================================
    \vspace{1em}
    \subsection{Explicación}

        Ok, parece que vayamos hecho una soberana estupidez, de analisis, pero dejame explicar
        por un lado ¿Que es lo peor que puede pasar?

        Pues nada, osea, que sin importar que pase tienes el mismo costo, operaciones elementales,
        sin importar que tan grandes sean $a, b, c$ su comparación no cuesta diferente.

        Por lo tanto tenemos que este es un algoritmo de ordenamiento sencillo de $O(1)$, el único
        algoritmo de ordenamiento que he visto así.





% ============================================================
% ===============    ALGORITMO  12 - 15      =================
% ============================================================
\clearpage
\section{Algoritmo 12 - 15}


    % ============================================================
    % =================          ALGORITMOS      =================
    % ============================================================
    \subsection{Análisis Visual}

        \begin{figure}[h]
            \centering
            \includegraphics[width=0.95\textwidth]{Algoritmo12}
            \caption{Análisis Visual del Algoritmo}
        \end{figure}

    % ============================================================
    % ============          EXPLICACION          =================
    % ============================================================
    \vspace{1em}
    \subsection{Explicación}

        Vamos de adentro hacia afuera, por un lado, primero, en el peor de los casos
        nuestro algoritmo siempre entrará dentro del if, dentro solo tenemos operaciones
        básicas, por lo tanto incluso si estamos lo que esta dentro de segundo for tiene un costo
        constante, es decir $O(1)$.

        Ahora este código constante se hará primero desde $0$ hasta $n$, luego de $1$ hasta $n$, y 
        así, por lo tanto podemos ver que si lo ordenamos al reves tenemos que se ejecutará $\sum_{i=1}^n i$
        veces pero lo lindo es que ese resultado es muy conocido, entonces entre los dos fors el código
        constante se hará $\frac{n(n+1)}{2}$, que si aplicamos el análisis sintótico tenemos que tiene
        un costo de $O(n^2)$ al final y al cabo.



\end{document}