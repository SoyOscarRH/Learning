

% *****************************************************************************************************
% ****************************              SIMPLE EXAMPLE             ********************************
% *****************************************************************************************************

% =======================================================
% =======         HEADER FOR DOCUMENT        ============
% =======================================================
    
    % *********  SPECIFIC FOR THIS BOOK  ********
    \def\ProjectAuthorLink{https://github.com/CompilandoConocimiento}
    \def\ProjectNameLink{\ProjectAuthorLink/LibroProbabilidad}    
    

    % *********   DOCUMENT ITSELF   **************
    \documentclass[12pt, fleqn]{article}                            %Type of doc and size of font and left equations
    \usepackage[margin=1.2in]{geometry}                             %Margins and Geometry pacakge
    \usepackage{ifthen}                                             %Allow simple programming using if - then
    \usepackage[hidelinks]{hyperref}                                %Allow to create hiperlinks and Fuck Firefox
    \usepackage{pdfpages}                                           %Allow us 'import' PDF's
    \hypersetup{pageanchor=false}                                   %Solve 'double page 1' warnings in build :v
    \setlength{\parindent}{0pt}                                     %Eliminate ugly indentation
    \author{Oscar Andrés Rosas}                                     %Who I am

    % *********   LANGUAJE    *****************
    \usepackage[spanish]{babel}                                     %Please allow me to type in spanish
    \usepackage[utf8]{inputenc}                                     %Lets use UFT-8
    \usepackage[T1]{fontenc}                                        %Allow for better font support
    \usepackage{textcmds}                                           %Allow us to use quoutes
    \usepackage{changepage}                                         %Allow us to use identate paragraphs
    \usepackage{anyfontsize}                                        %All the sizes for fonts wiiiii!

    % *********   MATH AND HIS STYLE  *********
    \usepackage{ntheorem, amsmath, amssymb, amsfonts}               %All fucking math, I want all!
    \usepackage{mathrsfs, mathtools, empheq}                        %All fucking math, I want all!
    \usepackage{cancel}                                             %Negate symbol
    \usepackage{centernot}                                          %Allow me to negate a symbol
    \decimalpoint                                                   %Use decimal point

    % *********   GRAPHICS AND IMAGES *********
    \usepackage{graphicx}                                           %Allow to create graphics
    \usepackage{float}                                              %For images
    \usepackage{wrapfig}                                            %Allow to create images
    \graphicspath{ {Graphics/} }                                    %Where are the images :D

    % *********   LISTS AND TABLES ***********
    \usepackage{listings, listingsutf8}                             %We will be using code here
    \usepackage[inline]{enumitem}                                   %We will need to enumarate
    \usepackage{tasks}                                              %Horizontal lists
    \usepackage{longtable}                                          %Lets make tables awesome
    \usepackage{booktabs}                                           %Lets make tables awesome
    \usepackage{tabularx}                                           %Lets make tables awesome
    \usepackage{multirow}                                           %Lets make tables awesome
    \usepackage{multicol}                                           %Create multicolumns

    % *********   REMOVE SOME ERRORS **********
    \hbadness=10000                                                 %Ignore \vbox and \hbox warings
    \hfuzz=\maxdimen\newdimen\hfuzz                                 %Ignore \vbox and \hbox warings

    % *********   HEADERS AND FOOTERS ********
    \usepackage{fancyhdr}                                           %Lets make awesome headers/footers
    \pagestyle{fancy}                                               %Lets make awesome headers/footers
    \setlength{\headheight}{16pt}                                   %Top line
    \setlength{\parskip}{0.5em}                                     %Top line
    \renewcommand{\footrulewidth}{0.5pt}                            %Bottom line


    \lhead{                                                         %Left Header
        \hyperlink{section.\arabic{section}}                        %Make a link to the current chapter
        {\normalsize{\textsc{\nouppercase{\leftmark}}}}             %And fot it put the name
    }

    \rhead{                                                         %Right Header
        \hyperlink{section.\arabic{section}.\arabic{subsection}}    %Make a link to the current chapter
            {\footnotesize{\textsc{\nouppercase{\rightmark}}}}      %And fot it put the name
    }

    \rfoot{\textsc{\small{\hyperref[sec:Index]{Ve al Índice}}}}     %This will always be a footer  

    \fancyfoot[L]{                                                  %Algoritm for a changing footer
        \ifthenelse{\isodd{\value{page}}}                           %IF ODD PAGE:
            {\href{https://SoyOscarRH.github.io/}                   %DO THIS:
                {\footnotesize                                      %Send the page
                    {\textsc{Oscar Andrés Rosas}}}}                 %Send the page
            {\href{https://compilandoconocimiento.com}              %ELSE DO THIS: 
                {\footnotesize                                      %Send the author
                    {\textsc{Compilando Conocimiento}}}}            %Send the author
    }
    
    
% =======================================================
% ===================   COMMANDS    =====================
% =======================================================

    % =========================================
    % =======   NEW ENVIRONMENTS   ============
    % =========================================
    \newenvironment{Indentation}[1][0.75em]                         %Use: \begin{Inde...}[Num]...\end{Inde...}
        {\begin{adjustwidth}{#1}{}}                                 %If you dont put nothing i will use 0.75 em
        {\end{adjustwidth}}                                         %This indentate a paragraph
    
    \newenvironment{SmallIndentation}[1][0.75em]                    %Use: The same that we upper one, just 
        {\begin{adjustwidth}{#1}{}\begin{footnotesize}}             %footnotesize size of letter by default
        {\end{footnotesize}\end{adjustwidth}}                       %that's it
    
    \def \Eq {equation}                                             %Stupid Visual studio error
    \newenvironment{MultiLineEquation}[1]                           %Use: To create MultiLine equations
        {\begin{\Eq}\begin{alignedat}{#1}}                          %Use: \begin{Multi..}{Num. de Columnas}
        {\end{alignedat}\end{\Eq}}                                  %And.. that's it!
    
    \newenvironment{MultiLineEquation*}[1]                          %Use: To create MultiLine equations
        {\begin{\Eq*}\begin{alignedat}{#1}}                         %Use: \begin{Multi..}{Num. de Columnas}
        {\end{alignedat}\end{\Eq*}}                                 %And.. that's it!
    

    % =========================================
    % == GENERAL TEXT & SYMBOLS ENVIRONMENTS ==
    % =========================================
    
    % =====  TEXT  ======================
    \newcommand \Quote              {\qq}                           %Use: \Quote to use quotes
    \newcommand \Over               {\overline}                     %Use: \Bar to use just for short
    \newcommand \ForceNewLine       {$\Space$\\}                    %Use it in theorems for example
    \newcommand \ForceColumnBreak   {\vfill\null\columnbreak}       %Use only in multicols

    % =====  SPACES  ====================
    \DeclareMathOperator \Space     {\quad}                         %Use: \Space for a cool mega space
    \DeclareMathOperator \MegaSpace {\quad \quad}                   %Use: \MegaSpace for a cool mega mega space
    \DeclareMathOperator \MiniSpace {\;}                            %Use: \Space for a cool mini space
    
    % =====  MATH TEXT  =================
    \newcommand \Such           {\MiniSpace | \MiniSpace}           %Use: \Such like in sets
    \newcommand \Also           {\MiniSpace \text{y} \MiniSpace}    %Use: \Also so it's look cool
    \newcommand \Remember[1]    {\Space\text{\scriptsize{#1}}}      %Use: \Remember so it's look cool
    
    % =====  THEOREMS: IN SPANISH :0  ===
    \newtheorem{Theorem}        {Teorema}[section]                  %Use: \begin{Theorem}[Name]\label{Nombre}...
    \newtheorem{Corollary}      {Colorario}[Theorem]                %Use: \begin{Corollary}[Name]\label{Nombre}...
    \newtheorem{Lemma}[Theorem] {Lemma}                             %Use: \begin{Lemma}[Name]\label{Nombre}...
    \newtheorem{Definition}     {Definición}[section]               %Use: \begin{Definition}[Name]\label{Nombre}...
    \theoremstyle{break}                                            %THEOREMS START 1 SPACE AFTER Fuck!

    % =====  LOGIC  =====================
    \newcommand \lIff    {\leftrightarrow}                          %Use: \lIff for logic iff
    \newcommand \lEqual  {\MiniSpace \Leftrightarrow \MiniSpace}    %Use: \lEqual for a logic double arrow
    \newcommand \lInfire {\MiniSpace \Rightarrow \MiniSpace}        %Use: \lInfire for a logic infire
    \newcommand \lLongTo {\longrightarrow}                          %Use: \lLongTo for a long arrow

    % =====  FAMOUS SETS  ===============
    \DeclareMathOperator \Naturals     {\mathbb{N}}                 %Use: \Naturals por Notation
    \DeclareMathOperator \Primes       {\mathbb{P}}                 %Use: \Primes por Notation
    \DeclareMathOperator \Integers     {\mathbb{Z}}                 %Use: \Integers por Notation
    \DeclareMathOperator \Racionals    {\mathbb{Q}}                 %Use: \Racionals por Notation
    \DeclareMathOperator \Reals        {\mathbb{R}}                 %Use: \Reals por Notation
    \DeclareMathOperator \Complexs     {\mathbb{C}}                 %Use: \Complex por Notation
    \DeclareMathOperator \GenericField {\mathbb{F}}                 %Use: \GenericField por Notation
    \DeclareMathOperator \VectorSet    {\mathbb{V}}                 %Use: \VectorSet por Notation
    \DeclareMathOperator \SubVectorSet {\mathbb{W}}                 %Use: \SubVectorSet por Notation
    \DeclareMathOperator \Polynomials  {\mathbb{P}}                 %Use: \Polynomials por Notation
    \DeclareMathOperator \VectorSpace  {\VectorSet_{\GenericField}} %Use: \VectorSpace por Notation
    \DeclareMathOperator \LinealTransformation {\mathcal{T}}        %Use: \LinealTransformation for a cool T
    \DeclareMathOperator \LinTrans      {\mathcal{T}}               %Use: \LinTrans for a cool T
    \DeclareMathOperator \Laplace       {\mathcal{L}}               %Use: \LinTrans for a cool T

    % =====  CONTAINERS   ===============
    \newcommand{\Set}[1]            {\left\{ \; #1 \; \right\}}     %Use: \Set {Info} for INTELLIGENT space 
    \newcommand{\bigSet}[1]         {\big\{  \; #1 \; \big\}}       %Use: \bigSet  {Info} for space 
    \newcommand{\BigSet}[1]         {\Big\{  \; #1 \; \Big\}}       %Use: \BigSet  {Info} for space 
    \newcommand{\biggSet}[1]        {\bigg\{ \; #1 \; \bigg\}}      %Use: \biggSet {Info} for space 
    \newcommand{\BiggSet}[1]        {\Bigg\{ \; #1 \; \Bigg\}}      %Use: \BiggSet {Info} for space 
        
    \newcommand{\Wrap}[1]           {\left( #1 \right)}             %Use: \Wrap {Info} for INTELLIGENT space
    \newcommand{\bigWrap}[1]        {\big( \; #1 \; \big)}          %Use: \bigBrackets  {Info} for space 
    \newcommand{\BigWrap}[1]        {\Big( \; #1 \; \Big)}          %Use: \BigBrackets  {Info} for space 
    \newcommand{\biggWrap}[1]       {\bigg( \; #1 \; \bigg)}        %Use: \biggBrackets {Info} for space 
    \newcommand{\BiggWrap}[1]       {\Bigg( \; #1 \; \Bigg)}        %Use: \BiggBrackets {Info} for space 

    \newcommand{\Brackets}[1]       {\left[ #1 \right]}             %Use: \Brackets {Info} for INTELLIGENT space
    \newcommand{\bigBrackets}[1]    {\big[ \; #1 \; \big]}          %Use: \bigBrackets  {Info} for space 
    \newcommand{\BigBrackets}[1]    {\Big[ \; #1 \; \Big]}          %Use: \BigBrackets  {Info} for space 
    \newcommand{\biggBrackets}[1]   {\bigg[ \; #1 \; \bigg]}        %Use: \biggBrackets {Info} for space 
    \newcommand{\BiggBrackets}[1]   {\Bigg[ \; #1 \; \Bigg]}        %Use: \BiggBrackets {Info} for space 

    \newcommand{\Generate}[1]   {\left\langle #1 \right\rangle}     %Use: \Generate {Info} <>
    \newcommand{\Floor}[1]      {\left \lfloor #1 \right \rfloor}   %Use: \Floor {Info} for floor 
    \newcommand{\Ceil}[1]       {\left \lceil #1 \right \rceil }    %Use: \Ceil {Info} for ceil
    
    % =====  BETTERS MATH COMMANDS   =====
    \newcommand{\pfrac}[2]      {\Wrap{\dfrac{#1}{#2}}}             %Use: Put fractions in parentesis

    % =========================================
    % ====   LINEAL ALGEBRA & VECTORS    ======
    % =========================================

    % ===== UNIT VECTORS  ================
    \newcommand{\hati}      {\hat{\imath}}                           %Use: \hati for unit vector    
    \newcommand{\hatj}      {\hat{\jmath}}                           %Use: \hatj for unit vector    
    \newcommand{\hatk}      {\hat{k}}                                %Use: \hatk for unit vector

    % ===== MAGNITUDE  ===================
    \newcommand{\abs}[1]    {\left\lvert #1 \right\lvert}           %Use: \abs{expression} for |x|
    \newcommand{\Abs}[1]    {\left\lVert #1 \right\lVert}           %Use: \Abs{expression} for ||x||
    \newcommand{\Mag}[1]    {\left| #1 \right|}                     %Use: \Mag {Info} 
    
    \newcommand{\bVec}[1]   {\mathbf{#1}}                           %Use for bold type of vector
    \newcommand{\lVec}[1]   {\overrightarrow{#1}}                   %Use for a long arrow over a vector
    \newcommand{\uVec}[1]   {\mathbf{\hat{#1}}}                     %Use: Unitary Vector Example: $\uVec{i}

    % ===== FN LINEAL TRANSFORMATION  ====
    \newcommand{\FnLinTrans}[1]{\mathcal{T}\Wrap{#1}}               %Use: \FnLinTrans for a cool T
    \newcommand{\VecLinTrans}[1]{\mathcal{T}\pVector{#1}}           %Use: \LinTrans for a cool T
    \newcommand{\FnLinealTransformation}[1]{\mathcal{T}\Wrap{#1}}   %Use: \FnLinealTransformation

    % ===== ALL FOR DOT PRODUCT  =========
    \makeatletter                                                   %WTF! IS THIS
    \newcommand*\dotP{\mathpalette\dotP@{.5}}                       %Use: \dotP for dot product
    \newcommand*\dotP@[2] {\mathbin {                               %WTF! IS THIS            
        \vcenter{\hbox{\scalebox{#2}{$\m@th#1\bullet$}}}}           %WTF! IS THIS
    }                                                               %WTF! IS THIS
    \makeatother                                                    %WTF! IS THIS

    % === WRAPPERS FOR COLUMN VECTOR ===
    \newcommand{\pVector}[1]                                        %Use: \pVector {Matrix Notation} use parentesis
        { \ensuremath{\begin{pmatrix}#1\end{pmatrix}} }             %Example: \pVector{a\\b\\c} or \pVector{a&b&c} 
    \newcommand{\lVector}[1]                                        %Use: \lVector {Matrix Notation} use a abs 
        { \ensuremath{\begin{vmatrix}#1\end{vmatrix}} }             %Example: \lVector{a\\b\\c} or \lVector{a&b&c} 
    \newcommand{\bVector}[1]                                        %Use: \bVector {Matrix Notation} use a brackets 
        { \ensuremath{\begin{bmatrix}#1\end{bmatrix}} }             %Example: \bVector{a\\b\\c} or \bVector{a&b&c} 
    \newcommand{\Vector}[1]                                         %Use: \Vector {Matrix Notation} no parentesis
        { \ensuremath{\begin{matrix}#1\end{matrix}} }               %Example: \Vector{a\\b\\c} or \Vector{a&b&c}

    % === MAKE MATRIX BETTER  =========
    \makeatletter                                                   %Example: \begin{matrix}[cc|c]
    \renewcommand*\env@matrix[1][*\c@MaxMatrixCols c] {             %WTF! IS THIS
        \hskip -\arraycolsep                                        %WTF! IS THIS
        \let\@ifnextchar\new@ifnextchar                             %WTF! IS THIS
        \array{#1}                                                  %WTF! IS THIS
    }                                                               %WTF! IS THIS
    \makeatother                                                    %WTF! IS THIS

    % =========================================
    % =======   FAMOUS FUNCTIONS   ============
    % =========================================

    % == TRIGONOMETRIC FUNCTIONS  ====
    \newcommand{\Cos}[1] {\cos\Wrap{#1}}                            %Simple wrappers
    \newcommand{\Sin}[1] {\sin\Wrap{#1}}                            %Simple wrappers
    \newcommand{\Tan}[1] {tan\Wrap{#1}}                             %Simple wrappers
    
    \newcommand{\Sec}[1] {sec\Wrap{#1}}                             %Simple wrappers
    \newcommand{\Csc}[1] {csc\Wrap{#1}}                             %Simple wrappers
    \newcommand{\Cot}[1] {cot\Wrap{#1}}                             %Simple wrappers

    % === COMPLEX ANALYSIS TRIG ======
    \newcommand \Cis[1]  {\Cos{#1} + i \Sin{#1}}                    %Use: \Cis for cos(x) + i sin(x)
    \newcommand \pCis[1] {\Wrap{\Cis{#1}}}                          %Use: \pCis for the same with parantesis
    \newcommand \bCis[1] {\Brackets{\Cis{#1}}}                      %Use: \bCis for the same with Brackets


    % =========================================
    % ===========     CALCULUS     ============
    % =========================================

    % ====== TRANSFORMS =============
    \newcommand{\FourierT}[1]   {\mathscr{F} \left\{ #1 \right\} }  %Use: \FourierT {Funtion}
    \newcommand{\InvFourierT}[1]{\mathscr{F}^{-1}\left\{#1\right\}} %Use: \InvFourierT {Funtion}

    % ====== DERIVATIVES ============
    \newcommand \MiniDerivate[1][x]   {\dfrac{d}{d #1}}             %Use: \MiniDerivate[var] for simple use [var]
    \newcommand \Derivate[2]          {\dfrac{d \; #1}{d #2}}       %Use: \Derivate [f(x)][x]
    \newcommand \MiniUpperDerivate[2] {\dfrac{d^{#2}}{d#1^{#2}}}    %Mini Derivate High Orden Derivate -- [x][pow]
    \newcommand \UpperDerivate[3] {\dfrac{d^{#3} \; #1}{d#2^{#3}}}  %Complete High Orden Derivate -- [f(x)][x][pow]
    
    \newcommand \MiniPartial[1][x] {\dfrac{\partial}{\partial #1}}  %Use: \MiniDerivate for simple use [var]
    \newcommand \Partial[2] {\dfrac{\partial \; #1}{\partial #2}}   %Complete Partial Derivate -- [f(x)][x]
    \newcommand \MiniUpperPartial[2]                                %Mini Derivate High Orden Derivate -- [x][pow] 
        {\dfrac{\partial^{#2}}{\partial #1^{#2}}}                   %Mini Derivate High Orden Derivate
    \newcommand \UpperPartial[3]                                    %Complete High Orden Derivate -- [f(x)][x][pow]
        {\dfrac{\partial^{#3} \; #1}{\partial#2^{#3}}}              %Use: \UpperDerivate for simple use

    \DeclareMathOperator \Evaluate  {\Big|}                         %Use: \Evaluate por Notation

    % ====== INTEGRALS ============
    \newcommand{\inftyInt} {\int_{-\infty}^{\infty}}                %Use: \inftyInt for simple integrants
    
        
% =======================================================
% ===========      COLOR: MATERIAL DESIGN     ===========
% =======================================================

    % =====  COLORS ==================
    \definecolor{RedMD}{HTML}{F44336}                               %Use: Color :D        
    \definecolor{Red100MD}{HTML}{FFCDD2}                            %Use: Color :D        
    \definecolor{Red200MD}{HTML}{EF9A9A}                            %Use: Color :D        
    \definecolor{Red300MD}{HTML}{E57373}                            %Use: Color :D        
    \definecolor{Red700MD}{HTML}{D32F2F}                            %Use: Color :D 

    \definecolor{PurpleMD}{HTML}{9C27B0}                            %Use: Color :D        
    \definecolor{Purple100MD}{HTML}{E1BEE7}                         %Use: Color :D        
    \definecolor{Purple200MD}{HTML}{EF9A9A}                         %Use: Color :D        
    \definecolor{Purple300MD}{HTML}{BA68C8}                         %Use: Color :D        
    \definecolor{Purple700MD}{HTML}{7B1FA2}                         %Use: Color :D 

    \definecolor{IndigoMD}{HTML}{3F51B5}                            %Use: Color :D        
    \definecolor{Indigo100MD}{HTML}{C5CAE9}                         %Use: Color :D        
    \definecolor{Indigo200MD}{HTML}{9FA8DA}                         %Use: Color :D        
    \definecolor{Indigo300MD}{HTML}{7986CB}                         %Use: Color :D        
    \definecolor{Indigo700MD}{HTML}{303F9F}                         %Use: Color :D 

    \definecolor{BlueMD}{HTML}{2196F3}                              %Use: Color :D        
    \definecolor{Blue100MD}{HTML}{BBDEFB}                           %Use: Color :D        
    \definecolor{Blue200MD}{HTML}{90CAF9}                           %Use: Color :D        
    \definecolor{Blue300MD}{HTML}{64B5F6}                           %Use: Color :D        
    \definecolor{Blue700MD}{HTML}{1976D2}                           %Use: Color :D        
    \definecolor{Blue900MD}{HTML}{0D47A1}                           %Use: Color :D  

    \definecolor{CyanMD}{HTML}{00BCD4}                              %Use: Color :D        
    \definecolor{Cyan100MD}{HTML}{B2EBF2}                           %Use: Color :D        
    \definecolor{Cyan200MD}{HTML}{80DEEA}                           %Use: Color :D        
    \definecolor{Cyan300MD}{HTML}{4DD0E1}                           %Use: Color :D        
    \definecolor{Cyan700MD}{HTML}{0097A7}                           %Use: Color :D        
    \definecolor{Cyan900MD}{HTML}{006064}                           %Use: Color :D 

    \definecolor{TealMD}{HTML}{009688}                              %Use: Color :D        
    \definecolor{Teal100MD}{HTML}{B2DFDB}                           %Use: Color :D        
    \definecolor{Teal200MD}{HTML}{80CBC4}                           %Use: Color :D        
    \definecolor{Teal300MD}{HTML}{4DB6AC}                           %Use: Color :D        
    \definecolor{Teal700MD}{HTML}{00796B}                           %Use: Color :D        
    \definecolor{Teal900MD}{HTML}{004D40}                           %Use: Color :D 

    \definecolor{GreenMD}{HTML}{4CAF50}                             %Use: Color :D        
    \definecolor{Green100MD}{HTML}{C8E6C9}                          %Use: Color :D        
    \definecolor{Green200MD}{HTML}{A5D6A7}                          %Use: Color :D        
    \definecolor{Green300MD}{HTML}{81C784}                          %Use: Color :D        
    \definecolor{Green700MD}{HTML}{388E3C}                          %Use: Color :D        
    \definecolor{Green900MD}{HTML}{1B5E20}                          %Use: Color :D

    \definecolor{AmberMD}{HTML}{FFC107}                             %Use: Color :D        
    \definecolor{Amber100MD}{HTML}{FFECB3}                          %Use: Color :D        
    \definecolor{Amber200MD}{HTML}{FFE082}                          %Use: Color :D        
    \definecolor{Amber300MD}{HTML}{FFD54F}                          %Use: Color :D        
    \definecolor{Amber700MD}{HTML}{FFA000}                          %Use: Color :D        
    \definecolor{Amber900MD}{HTML}{FF6F00}                          %Use: Color :D

    \definecolor{BlueGreyMD}{HTML}{607D8B}                          %Use: Color :D        
    \definecolor{BlueGrey100MD}{HTML}{CFD8DC}                       %Use: Color :D        
    \definecolor{BlueGrey200MD}{HTML}{B0BEC5}                       %Use: Color :D        
    \definecolor{BlueGrey300MD}{HTML}{90A4AE}                       %Use: Color :D        
    \definecolor{BlueGrey700MD}{HTML}{455A64}                       %Use: Color :D        
    \definecolor{BlueGrey900MD}{HTML}{263238}                       %Use: Color :D        

    \definecolor{DeepPurpleMD}{HTML}{673AB7}                        %Use: Color :D

    % =====  ENVIRONMENT ==============
    \newcommand{\Color}[2]{\textcolor{#1}{#2}}                      %Simple color environment
    \newenvironment{ColorText}[1]                                   %Use: \begin{ColorText}
        { \leavevmode\color{#1}\ignorespaces }                      %That's is!


% =======================================================
% ===========           CODE EDITING          ===========
% =======================================================

    % =====  CODE EDITOR =============
    \lstdefinestyle{CompilandoStyle} {                              %This is Code Style
        backgroundcolor     = \color{BlueGrey900MD},                %Background Color  
        basicstyle          = \tiny\color{white},                   %Style of text
        commentstyle        = \color{BlueGrey200MD},                %Comment style
        stringstyle         = \color{Green300MD},                   %String style
        keywordstyle        = \color{Blue300MD},                    %keywords style
        numberstyle         = \tiny\color{TealMD},                  %Size of a number
        frame               = shadowbox,                            %Adds a frame around the code
        breakatwhitespace   = true,                                 %Style   
        breaklines          = true,                                 %Style   
        showstringspaces    = false,                                %Hate those spaces                  
        breaklines          = true,                                 %Style                   
        keepspaces          = true,                                 %Style                   
        numbers             = left,                                 %Style                   
        numbersep           = 10pt,                                 %Style 
        xleftmargin         = \parindent,                           %Style 
        tabsize             = 4,                                    %Style
        inputencoding       = utf8/latin1                           %Allow me to use special chars
    }

    % =====  CODE EDITOR =============
    \lstdefinestyle{CompilandoStylePurity} {                        %This is Code Style
        backgroundcolor     = \color{white},                        %Background Color  
        basicstyle          = \tiny\color{BlueGrey900MD},           %Style of text
        commentstyle        = \color{Green300MD},                   %Comment style
        stringstyle         = \color{Teal700MD},                    %String style
        keywordstyle        = \color{Blue700MD},                    %keywords style
        numberstyle         = \tiny\color{TealMD},                  %Size of a number
        frame               = none,                                 %Adds a frame around the code
        breakatwhitespace   = true,                                 %Style   
        breaklines          = true,                                 %Style   
        showstringspaces    = false,                                %Hate those spaces                  
        breaklines          = true,                                 %Style                   
        keepspaces          = true,                                 %Style                   
        numbers             = left,                                 %Style                   
        numbersep           = 11pt,                                 %Style 
        xleftmargin         = \parindent,                           %Style 
        tabsize             = 4,                                    %Style
        inputencoding       = utf8/latin1                           %Allow me to use special chars
    }
 
    \lstset{style = CompilandoStyle}                                %Use this style
    
    
    
    

% =====================================================
% ============        COVER PAGE       ================
% =====================================================
\begin{document}
\begin{titlepage}
    
    % ============ TITLE PAGE STYLE  ================
    \definecolor{TitlePageColor}{cmyk}{1,.60,0,.40}                 %Simple colors
    \definecolor{ColorSubtext}{cmyk}{1,.50,0,.10}                   %Simple colors
    \newgeometry{left=0.25\textwidth}                               %Defines an Offset
    \pagecolor{TitlePageColor}                                      %Make it this Color to page
    \color{white}                                                   %General things should be white

    % ===== MAKE SOME SPACE =========
    \vspace                                                         %Give some space
    \baselineskip                                                   %But we need this to up command

    % ============ NAME OF THE PROJECT  ============
    \makebox[0pt][l]{\rule{1.3\textwidth}{3pt}}                     %Make a cool line
    
    \href{https://compilandoconocimiento.com}                       %Link to project
    {\textbf{\textsc{\Huge Estructuras Discretas}}}\\[2.7cm]      %Name of project   

    % ============ NAME OF THE BOOK  ===============
    \href{\ProjectNameLink}                                         %Link to Author
    {\fontsize{45}{52}\selectfont \textbf{Tarea 1}}\\[0.5cm]        %Name of the book
    \textcolor{ColorSubtext}{\textsc{\Huge Manejo general de \LaTeX}}     %Name of the general theme
    
    \vfill                                                          %Fill the space
    
    % ============ NAME OF THE AUTHOR  =============
    \href{\ProjectAuthorLink}                                       %Link to Author
    {\LARGE \textsf{Oscar Andrés Rosas Hernandez}}                  %Author

    % ===== MAKE SOME SPACE =========
    \vspace                                                         %Give some space
    \baselineskip                                                   %But we need this to up command
    
    {\large \textsf{Agosto 2018}}                                    %Date

\end{titlepage}


% =====================================================
% ==========      RESTORE TO DOCUMENT      ============
% =====================================================
\restoregeometry                                                    %Restores the geometry
\nopagecolor                                                        %Use to restore the color to white




% =====================================================
% ========                INDICE              =========
% =====================================================
\tableofcontents{}
\label{sec:Index}

\clearpage


% ===============================================
% ========              1                 =======
% ===============================================
\clearpage
\section{1}

    Sea la gramatica:
    \begin{itemize}
        \item $S ::= C$
        \item $C ::= ACB \Such d$
        \item $A ::= b \Such aaBB \Such aA$
        \item $B ::= dcd$
    \end{itemize}

    Ahora generemos 4 expresiones que tengan al menos 7 símbolos terminales:
    \begin{itemize}
        
        \item 
            \begin{MultiLineEquation*}{3}
                &S                                  \\
                &C                                  \\
                &A && B &&C                         \\
                &aaBB && B &&C                      \\
                &aaBB && dcd &&C                    \\
                &aaBB && dcd &&C                    \\
                &aaBB && dcd &&d                    \\
                &aadcddcd && dcd &&d
            \end{MultiLineEquation*}

        \item 
            \begin{MultiLineEquation*}{5}
                &S                                  \\
                &C                                  \\
                &A && B &&C                         \\
                &aaBB && B &&C                      \\
                &aaBB && dcd &&C                    \\
                &aaBB && dcd &&C                    \\
                &aaBB && dcd &&A &&B &&C            \\
                &aadcddcd && dcd &&aaBB && dcd & d  \\
                &aadcddcd && dcd &&aadcddcd && dcd & d  \\
            \end{MultiLineEquation*}

        \item 
            \begin{MultiLineEquation*}{5}
                &S                                  \\
                &C                                  \\
                &A && B &&C                         \\
                &aA && B &&C                      \\
                &aaaBB && dcd &&C                    \\
                &aaaBB && dcd &&C                    \\
                &aaaBB && dcd &&A &&B &&C            \\
                &aaadcddcd && dcd &&aaBB && dcd & d  \\
                &aaadcddcd && dcd &&aadcddcd && dcd & d  \\
            \end{MultiLineEquation*}

        \item 
            \begin{MultiLineEquation*}{5}
                &S                                  \\
                &C                                  \\
                &A && B &&C                         \\
                &aA && B &&C                      \\
                &aaaBB && dcd &&C                    \\
                &aaaBB && dcd &&C                    \\
                &aaaBB && dcd &&A &&B &&C            \\
                &aaaBB && dcd &&A &&aA &&C            \\
                &aaadcddcd && dcd &&aaaBB && dcd & d  \\
                &aaadcddcd && dcd &&aaadcddcd && dcd & d  \\
            \end{MultiLineEquation*}

    \end{itemize}




% ===============================================
% ========              2                 =======
% ===============================================
\clearpage
\section{2}

    Sea la cadena $aadcddcdabaadcddcddcddcddcd$

    Entonces veamos su derivación:
    \begin{MultiLineEquation*}{3}
        aa dcd dcd ab d dcd dcd     \\
        aa B dcd ab d dcd dcd       \\
        aa B B b d B B              \\
        aa B B A C B B              \\
        A A C B B                   \\
        A C B                       \\
        C
    \end{MultiLineEquation*}

    Ahora, por el otro lado veamos la cadena b).
    Sea la cadena $aadcddcdabaadcddcddcddcddcd$, intentemos su derivación:
    \begin{MultiLineEquation*}{3}
        aa dcd dcd a b aa dcd dcd dcd dcd dcd \\
        aa B B a b aa B B B B B                 \\
        aa B B a b A B B B                      \\
        aa B B a C A B B B                      \\
        A a C A B B B
    \end{MultiLineEquation*}    

    Por lo tanto, no, no esta bien formada.




% ===============================================
% ========            3                   =======
% ===============================================
\clearpage
\section{3}        

    Suponiendo que tengamos las 10 reglas que conocemos:
    \begin{itemize}
        \item $S ::= E$
        \item $E ::= var$
        \item $E ::= const$
        \item $E ::= \circ E$
        \item $E ::= E \ast E$
        \item $E ::= (E)$
    \end{itemize}

    Para crear expresiones como $min(2, a+b, \dots, 5)$

    Lo que nos falta es añadir las siguientes reglas:
    \begin{itemize}
        \item $E ::= \ast(X)$
        \item $X ::= E$
        \item $X ::= E, E$
    \end{itemize}


% ===============================================
% ========            10                  =======
% ===============================================
\section{4}
En las fotos


% ===============================================
% ========            5                   =======
% ===============================================
\vspace{1em}
\section{5}        

    \begin{itemize}
        \item $S ::= E$
        \item $E ::= 1$
        \item $E ::= 0$
        \item $E ::= 1E$
        \item $E ::= 0E$
    \end{itemize}


% ===============================================
% ========            6                   =======
% ===============================================
\clearpage
\section{6}        

    Proposiciones atomicas:
    \begin{itemize}
        \item El lago Ness esta en Irlanda
        \item Esta tarea fue hecha hace 5 minutos
        \item Una nebulosa planetaria existe en nuestro sistema solar
    \end{itemize}


    Proposiciones no atomicas:
    \begin{itemize}
        \item El lago Ness esta en Irlanda e Irlanda es un país
        \item Esta tarea fue hecha hace 5 minutos ó hace una hora
        \item Una nebulosa planetaria existe en nuestro sistema solar ó una estrella existe en el
        sistema solar
    \end{itemize}


% ===============================================
% ========            7                   =======
% ===============================================
\vspace{1em}
\section{7}        

    \begin{itemize}
        \item a. Proposición. $2 + 5 \neg 19$
        \item b. No
        \item c. Proposición. Para ningun entero positivo, n se cumple que $19340 = n * 17$
        \item d. Proposición. Audrey Meadows no fue la Alicia original en The Honeymooners
        \item e. No
        \item f. La frase 'hazlo de nuevo, Sam' no aparece en la pelicula casablanca
        \item g. Existe algun entero par mayor que cuatro que no es la suma de dos primos
        \item h. No
    \end{itemize}


% ===============================================
% ========            8                   =======
% ===============================================
\clearpage
\section{8}   

\begin{multicols}{2}


    Sea $p = r = F$ y $q = V$     

    \begin{itemize}
        
        \item a)
            \begin{MultiLineEquation*}{3}
                p \vee q                               \\
                F \vee V                                \\
                V
            \end{MultiLineEquation*}


        \item b)
            \begin{MultiLineEquation*}{3}
                \neg p \vee q                               \\
                \neg F \vee V                                \\
                V \vee V                                \\
                V
            \end{MultiLineEquation*}

        \item c)
            \begin{MultiLineEquation*}{3}
                \neg (p \vee q) \wedge \neg p               \\
                \neg (V) \wedge \neg p               \\
                F \wedge \neg p               \\
                F
            \end{MultiLineEquation*}

        \item d)
            \begin{MultiLineEquation*}{3}
                \neg p \vee \neg q                               \\
                \neg F \vee \neg q                                \\
                V \vee \neg q                                \\
                V
            \end{MultiLineEquation*}

        \item e)
            \begin{MultiLineEquation*}{3}
                \neg p \vee \neg (q \wedge r)                \\
                \neg F \vee \neg (q \wedge r)                \\
                V \vee \neg (q \wedge F)                \\
                V \vee \neg F                \\
                V \vee V                \\
                V
            \end{MultiLineEquation*}


        \item f)
            \begin{MultiLineEquation*}{3}
                (p \vee \neg r) \wedge \neg((q \vee r) \wedge \neg(r \vee p))  \\
                (p \vee V) \wedge \neg((q \vee r) \wedge \neg(r \vee p))        \\
                V \wedge \neg((q \vee r) \wedge \neg(r \vee p))                 \\
                V \wedge \neg(V \wedge \neg(r \vee p))                          \\
                V \wedge \neg(V \vee V)                                       \\
                V \wedge F                                                      \\
                F
            \end{MultiLineEquation*}
                    

    \end{itemize}

\end{multicols}



% ===============================================
% ========            10                  =======
% ===============================================
\clearpage
\section{9}
En las fotos

% ===============================================
% ========            10                  =======
% ===============================================
\section{10}
En las fotos


% ===============================================
% ========            10                  =======
% ===============================================
\clearpage
\section{11}

    Sea p: Hoy es lunes, q: Está lloviendo, r: Hace calor

    Entonces:
    \begin{itemize}
        \item $p \vee q$: Hoy es lunes ó está lloviendo
        \item $(p \wedge q) \wedge \neg(r \vee p)$: Hoy es lunes y está lloviendo ó no pasa que
        hace calor ó es lunes
        \item $(p \wedge (q \vee r)) \wedge (r \vee (q \vee p))$: 

        Hoy es lunes y pasa que esta lloviendo o hace calor, y , hace calor o pasa que esta lloviendo
        o hoy es lunes
        \item $\neg p \wedge (q \vee r)$: No es cierto que hoy es lunes y o esta lloviendo o hace calor   
    \end{itemize}


% ===============================================
% ========            10                  =======
% ===============================================
\section{12}

    No la violo, pongamos la ley como: No se permite que una persona $p \wedge q$.

    Donde p es tener mas de 3 perros y q es tener mas de 3 gatos.
    Ahora podemos que ver p es cierta pero q no, por lo tanto $p \wedge q$ es falso, por 
    lo tanto no rompio la ley.

% ===============================================
% ========            13                  =======
% ===============================================
\section{13}


    \begin{MultiLineEquation*}{3}
        P \lLongTo (Q \lLongTo R)
            &\lEqual \neg (Q \lLongTo R) \lLongTo \neg P   \\
            &\lEqual \neg (\neg Q \vee R) \lLongTo \neg P  \\
            &\lEqual (Q \wedge \neg R) \lLongTo \neg P     \\
            &\lEqual (\neg R \wedge Q) \lLongTo \neg P
    \end{MultiLineEquation*}


% ===============================================
% ========            14                  =======
% ===============================================
\clearpage
\section{14}

    Provemos por tabla de verdad: \\
    \begin{tabular}{|l|l|l|l|l|l|} 
        \hline
        p & q & r & (p $\lEqual$ q) &$\lEqual$  &r   \\ \hline
        0 & 0 & 0 &        1        &     0      &0   \\ \hline 
        0 & 0 & 1 &        1        &     1      &1   \\ \hline 
        0 & 1 & 0 &        0        &     1      &0   \\ \hline 
        0 & 1 & 1 &        0        &     0      &1   \\ \hline 
        1 & 0 & 0 &        0        &     1      &0   \\ \hline 
        1 & 0 & 1 &        0        &     0      &1   \\ \hline 
        1 & 1 & 0 &        1        &     0      &0   \\ \hline 
        1 & 1 & 1 &        1        &     1      &1   \\ \hline 
    \end{tabular}

    Y ve que es igual a: \\
        \begin{tabular}{|l|l|l|l|l|l|} 
        \hline
        p & q & r & p   & $\lEqual$   & $(q \lEqual r)$  \\ \hline
        0 & 0 & 0 & 0   &   0        &     1          \\ \hline 
        0 & 0 & 1 & 0   &   1        &     0          \\ \hline 
        0 & 1 & 0 & 0   &   1        &     0          \\ \hline 
        0 & 1 & 1 & 0   &   0        &     1          \\ \hline 
        1 & 0 & 0 & 1   &   1        &     1          \\ \hline 
        1 & 0 & 1 & 1   &   0        &     0          \\ \hline 
        1 & 1 & 0 & 1   &   0        &     0          \\ \hline 
        1 & 1 & 1 & 1   &   1        &     1          \\ \hline 
    \end{tabular}

    Por otro lado, veamos un contraejemplo para el otro problema, 
    nota que si $p = F$, $q = V$ y $r = F$ 
    entonces $(p \lInfire q) \lInfire r$ es falso pero $p \lInfire (q \lInfire r)$
    es verdadero, por lo tanto la implicación no es asociativa.




% ===============================================
% ========            14                  =======
% ===============================================
\clearpage
\section{15}

    Provemos por tabla de verdad:\\
    \begin{tabular}{|l|l|l|} 
        \hline
        p & q  & (p $\wedge$ q)   \\ \hline
        0 & 0  &        0         \\ \hline 
        0 & 1  &        0         \\ \hline 
        1 & 0  &        0         \\ \hline 
        1 & 1  &        1         \\ \hline 
    \end{tabular}

    y:\\
    \begin{tabular}{|l|l|l|} 
        \hline
        p & q  & (q $\wedge$ p)   \\ \hline
        0 & 0  &        0         \\ \hline 
        0 & 1  &        0         \\ \hline 
        1 & 0  &        0         \\ \hline 
        1 & 1  &        1         \\ \hline 
    \end{tabular}

  

    Por otro lado, veamos un contraejemplo para el otro problema, 
    nota que si $p = F$, $q = V$ entonces $p \lInfire q$ es verdadero
    pero $q \lInfire p$ es falso.





% ===============================================
% ========            16                  =======
% ===============================================
\section{16}

    \begin{itemize}
        \item 
            Pasemos de: $R \vee S \lInfire R \lInfire Q \wedge S \lInfire Q$
            pasa a ser:\\
            $( ( (R \vee S) \lInfire R) \lInfire (Q \wedge S) ) \lInfire Q$

        \item
            Pasemos de: $R \vee Q \lEqual \neg R \lInfire Q$
            pasa a ser:\\
            $(R \vee Q) \lEqual  ((\neg R) \lInfire Q)$    
    \end{itemize}

            
% ===============================================
% ========            17                  =======
% ===============================================
\section{17}    

El or exclusivo se peude imitar como $(p \vee q) \wedge \neg(p \wedge q)$
veamos una tabla de verdad:             \\

\begin{tabular}{|l|l|l|l|l|} 
    \hline
    p & q  & (p $\vee$ q)  &$\wedge$ &$\neg(p \wedge q)$   \\ \hline
    0 & 0  &        0    &    0   & 1     \\ \hline 
    0 & 1  &        1    &    1   & 1    \\ \hline 
    1 & 0  &        1    &    1   & 1     \\ \hline 
    1 & 1  &        1    &    0   & 0     \\ \hline 
\end{tabular}















                
\end{document}